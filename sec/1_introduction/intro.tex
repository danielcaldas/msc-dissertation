%% ---------------------------------------------- Chapter Introduction
In this chapter we do an introductory overview on the work being developed along this master's dissertation. This chapter presents the essential introductory topics. First we present the problem and context where this project is framed, then we expose the motivation, followed by the research hypothesis, which concisely describes the possible outcome of this project. Finally we list the goals of this project in a generic and simple way.

%% ---------------------------------------------- Context and Problem
\section{Context and Problem}

In the mid 1950s sociologists introduced the term \acrfull{sn}, that despite being a familiar term for today general public because of the \glspl{osn} platforms such as Facebook, Instagram or Twitter, it is a deeper and more mature concept. It was in the 2000s that much of the \glspl{osn} we know today start emerging, so it took at least ten years to people to adopt the concept and the new way of living, so today billions of people use these online platforms as channels for socializing, connect with each other and share their daily lives.\\
\indent From the user's point of view we may consider that all the platforms offer a microscopic perspective from within the network, people have a public profile, and they can visualize their friend's profile (this is a typical scenario that we observe today in the majority of the \glspl{osn}), and normally have access to a timeline that displays friends activity. The point is that, to the users of these online platforms, it is not provided a mean to visualize and analyze their network structure in a more abstract and generalized sense, where users are given the opportunity to observe their social network from a macroscopic perspective, and, with that, all the metrics for measuring nodes and relationships within the network.\\
\indent The problem that is being built in this section resides on general social structure observation and analyzes. This dissertation aims to fill the gap or struggle that online social networks users have in understand their network, how their relationships evolve along the time, what role they play within the network and how can they analyze and visualize their networks based on social properties such as mutual relationships, geographical position, personal tastes and preferences or hobbies.\\
\indent Within the big data challenges, social network data analysis might be one of the biggest demands that we face today, because besides of dealing with tremendous amounts of data, we are dealing with unstructured data. The unstructured data derives from the diversity of this platforms known as \glspl{osn}, and unstructured data adds complexity to the challenge of analyzing social networks data. The major challenges related with big data and unstructured data comes after the data extraction.

\subsection*{The steps for data analyzes and visualization}

Next we present the steps trough data extraction to data visualization, that generally represent the structure and flow of data analysis and visualization systems.

\begin{itemize}
\item Data extraction trough social media \glspl{api} or trough web crawlers (also known as web scrappers);
\item Saving data, and more importantly know what data to store; in order to have an efficient system that provides a good structure for data analysis, one needs
to selected data carefully;
\item What to do with the data, what applications the stored data may have, how can the system digest and transform data in order to make it useful or interesting for the end users;
\item How to present/show the transformed data, despite the science of visualization represents only a small part of the data scientist work, it has a huge impact on the end user, mainly when targeting a general audience.
\end{itemize}

%% ---------------------------------------------- Motivation
\section{Motivation}
As we see in the previous section, social media data analysis represents a major challenge for data scientists in every aspect, since the extraction all the way to the visualization. Despite representing a major technological challenge, social media data analyzes has an additional motivation, that is the massive daily usage in every country across the planet making \glspl{osn} an universal tool for communication, such as radio or television but with the technological flavor of the 21st century.\\
\indent \glspl{osn} as we will see along this dissertation, are today a \textit{"digital mineral"} in terms of exploration potential, we do not only pretend to have a generalist perspective of the analyzes of data that flows within this platforms, we will try when appropriate to demonstrate the most narrower applications as possible of analyzing social networks, this applications may go from health analyzes within social structures, to strategic marketing planning supported by the analysis of the already mentioned unstructured data.

%% ---------------------------------------------- Research Hypothesis
\section{Research Hypothesis}
% A thesis is a statement or theory that is put forward as a premise to be maintained or proved.
% e.g. "his central thesis is that psychological life is not part of the material world"
% A research hypothesis is the statement created by researchers when they speculate upon the outcome of a research or experiment.
% If _____[I do this] _____, then _____[this]_____ will happen." ...
With this master's dissertation, we aim to prove that a software tool may be designed and implemented in order to actually improve the analysis of social phenomena, allowing not only sociologists but also the public in general to explore with greater
detail the connections of individuals within a network, being \glspl{osn} the base of analysis for such a tool.

%% ---------------------------------------------- Goals
\section{Goals}

The main goal of this project is to build a useful software tool in the context of social network analysis. Along the process of building and investigating, the following are some of the goals that are also very important to achieve:

\begin{itemize}
\item Understand the theory of \gls{sn} in sociology;
\item Understand how \glspl{osn} came to such a massive use nowadays;
\item Perceive the roles of Online Social Networks in society and their potential applications in various fields;
\item Study and analyze the most used Online Social Networks, learn how to interact with those systems and how to learn and profit from them;
\item Design a system of analysis and visualization that matches the desired goals and requirements;
\item Explore new technologies and choose the appropriate tools to build the specified system;
\item Implement the system.
\end{itemize}

%% ---------------------------------------------- Document Structure
\section{Document Structure}
In this section we will preset how this document is structure, and in concisely explain what to expect in each of the following chapters.\\
\indent We start by exploring some theoretical background on \glspl{sn} in light of sociology. In Chapter 2 we present some of the history behind \glspl{sn}, and we review some of the fundamental concepts of \glspl{sn}.\\
\indent In Chapter 3 we explore \acrfull{osn}, this is the \textit{personification} of the \gls{sn} concept of the $21^{st}$. Primarily we present a top level overview on \glspl{osn}, where we present many of them and some metrics to compare them (such as number of active and registered users), then we provide again some historical background followed by the detailed analysis of some selected \glspl{osn}, also we talk about \glspl{osn} usage among Portuguese people and what impact \glspl{osn} had in a recent past and continue to have.\\
\indent In Chapter 4 we discuss a very broad theme \acrfull{sna}, in the scope of our project. We talk about \glspl{sna} basic concepts and metrics that are useful for network analysis as well as related scientific related areas. We do an overview on \glspl{sna} software tools and libraries.\\
\indent In Chapter 5 we present an architectural perspective of the project to develop along this master's thesis. We end this document with the conclusion, working plan and future work.
