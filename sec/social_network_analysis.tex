%% ---------------------------------------------- Chapter Introduction

% Social network analysis is the application of network theory to the modeling and analysis of social systems.
% it combine both tools for analyzing social relations and theory for explaining the structures that emerge from the social interactions.
%
% Of course the idea of studying societies as networks is not a new one but with the rise in computation and the
% emergence of a mass of new data sources, social network analysis is beginning to be applied to all type and
% scales of social systems from, international politics to local communities and everything in between.
%
% Traditionally when studying societies we think of them as composed of various types of individuals and organizations,
% we then proceed to analysis the properties to these social entities such as their age, occupation or population, and them ascribe quantitative value to them.
%
% This allows social science to use the formal mathematical language of statistical analyst to compare the values of
%  these properties and create categories such as low in come house holds or generation x, we then search for quasi
%  cause and effect relations that govern these values.
%
% This component-based analysis is a powerful method for describing social systems. Unfortunately though is fails to
% capture the most important feature of social reality that is the relations between individuals, statistical analysis
% present a picture of individuals and groups isolates from the nexus of social relations that given them context.
%
% Thus we can only get so far by studying the individual because when individuals interact and organize, the results
% can be greater than the simple sum of its parts, it is the relations between individuals that create the emergent
% property of social institutions and thus to understand these institutions we need to understand the networks of social relations that constitute them.
%
% Ever since the emergence of human beans we have been building \glspl{sn}, we live our lives embed in networks
% of relations, the shape of these structures and where we lie in them all effect our identity and perception of the world.
%
% A social network is a system made up of a set of social actors such as individuals or organizations and a set of
% ties between these actors that might be relations of friendship, work colleagues or family. Social network science
% then analyze empirical data and develops theories to explaining the patterns observed in these networks
%
% In so doing we can begin to ask questions about the degree of connectivity within a network, its over all structure,
% how fast something will diffuse and propagate through it or the Influence of a given node within the network. lets take some examples of this
%
% Social network analysis has been used to study the structure of influence within corporations, where traditionally
% we see organization of this kind as hierarchies, by modeling the actual flow of information and communication as a
% network we get a very different picture, where seemingly irrelevant employees within the hierarchy can in fact have significant influence within the network.
%
% Researcher also study innovation as a process of diffusion of new ideas across networks, where the oval structure
% to the network, its degree of connectivity, centralization or decentralization are a defining feature in the way
% that innovation spreads or fails to spread.
%
% Network dynamics, that is how networks evolve overtime is another important area of research, for example within Law
% enforcement agencies social network analysis is used to study the change in structure of terrorists groups to identify
% changing relations through which they are created, strengthened and dissolved?
%
% Social network analysis has also been used to study patterns of segregation and clustering within international politics
% and culture, by mapping out the beliefs and values of countries and cultures as networks we can identify where opinions and beliefs overlap or conflict.
%
% Social network analysis is a powerful new method we now have that allows us to convert often large and dense data sets
% into engaging visualization, that can quickly and effectively communicate the underlining dynamics within the system.
%
% By combine new discoveries in the mathematics of network theory, with new data sources and our sociological understanding,
% social network analysis is offering huge potential for a deeper, richer and more accurate understanding, of the complex social systems that make up our world.


% RATIONALE
% As the Web rapidly evolves, Web users are evolving with it. In an era of social
% connectedness, people are becoming increasingly enthusiastic about interacting,
% sharing, and collaborating through social networks, online communities, blogs,
% Wikis, and other online collaborative media. In recent years, this collective
% intelligence has spread to many different areas, with particular focus on fields
% related to everyday life such as commerce, tourism, education, and health,
% causing the size of the Web to expand exponentially.
%
% The distillation of knowledge from such a big amount of unstructured
% information, however, is an extremely difficult task, as the contents of today’s
% Web are perfectly suitable for human consumption, but remain hardly accessible
% to machines. The opportunity to capture the opinions of the general public about
% social events, political movements, company strategies, marketing campaigns, and
% product preferences has raised growing interest both within the scientific
% community, leading to many exciting open challenges, as well as in the business
% world, due to the remarkable benefits to be had from marketing and financial
% market prediction.
%
% Existing approaches to big social data analysis mainly rely on parts of text in
% which sentiment is explicitly expressed, e.g., through polarity terms or affect
% words (and their co-occurrence frequencies). However, opinions and sentiments
% are often conveyed implicitly through latent semantics, which make purely
% syntactical approaches ineffective. In this light, this Special Issue focuses on
% the introduction, presentation, and discussion of novel techniques that further
% develop and apply affective reasoning tools and techniques for big social data
% analysis. A key motivation for this Special Issue, in particular, is to explore
% the adoption of novel affective reasoning frameworks and cognitive learning
% systems to go beyond a mere word-level analysis of natural language text and
% provide novel concept-level tools and techniques that allow a more efficient
% passage from (unstructured) natural language to (structured) machine-processable
% affective data, in potentially any domain.

\gls{sna} is the study of how people are connected to each other, basically it studies a set of relations among a set of entities,
these entities may be individuals, organizations, or even countries.\\\\
\indent The common analysis procedure consists in mapping the network and then create metrics to
characterize the network. Then one tries to figure what is the structure of the network and why does
it have that structure. \gls{sna} is also about look at the individuals inside the network and where are those individuals located.

\section{Fundamental Concepts for Network Analysis}

The concepts listed below are of key importance to understand \gls{sna}.\cite{wasserman1994social}

\begin{itemize}
    \item \emph{Actor} - \gls{sna} is concerned with understanding the linkages among social entities and the implications of these linkages, these social entities are described as actors. Actors are are discrete individual, corporate, or collective social units.
    \item \emph{Relational Tie} - Actors are linked to one another trough \textit{social ties}. The type of ties may be extensive, and it describes the nature of the connection. Some example of ties:
        \begin{itemize}
            \item \textbf{Evaluation} of one person by another;
            \item \textbf{Transference} of resources (business transactions);
            \item \textbf{Association} (to social event or cause);
            \item \textbf{Behavioural} interactions (communicating);
            \item \textbf{Moving} between places or statuses (migration, social or physical mobility);
            \item Others may be: physical connection (roads, rivers), formal relations (authority), biological relationship;
        \end{itemize}
    \item \emph{Dyad} - The most basic relationship that can be established is a dyad, a connection between two actors.
    \item \emph{Triad} - A relation established between three actors. Many studies included breaking \glspl{sn} down to small groups (triads), this allowed a more clear conclusion about the transitivity of the connections.
    \item \emph{Subgroup} - It defines any subset of actors in a \gls{sn} (conceptually, subgroups come after dyads and triads).
    \item \emph{Group} - A finite set of actors who for conceptual, theoretical or empirical reasons are treated as a finite set of individuals in which network measurements are made.
    \item \emph{Relation} - A collection of ties of a specific kind among members of a group is called a \textbf{relation} (e.g. a connection in \textit{LinkedIn} is a relation while evaluating our connections of sending them messages are ties).
    \item \emph{\gls{sn}} - With the definitions of actor, group and relation, a \gls{sn} consists of a finite set or sets of actors and
    and the relation or relations defined on them. The presence of relation information is critical and defining feature of a \gls{sn}.
\end{itemize}

\section{Network Analysis}
\subsection{Scientific Background}
\subsection{Graphs Theory}
\subsection{Statistics}
\subsection{...}
\subsection{Power Law}
\subsection{Centrality Measures}
\subsection{Community Detection}
\subsection{Spread of Information}
\subsection{Link Analysis}
\subsection{...}
\section{Six Degrees of Separation}
% Small World Problem, Stanley Milgram's Experiment

\section{Network Visualization}
% It's a science by itself. REF ao documento STAR report on Dynamic Graph Visualization

\section{Real World Applications}
% What SNAs are used for.

\section{Social Network Analysis Software}
