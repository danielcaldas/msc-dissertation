\section{Portuguese People and Online Social Networks}
From Table \ref{table:osns}, we get a good overview on \glspl{osn} usage among modern society. In this section we do a deep exploration of the most adopted \glspl{osn} by portuguese citizens,
and get to compare then with the more global scenario presented in Table \ref{table:osns}, also, other interesting facts will be revealed where appropriate.\\
\indent A recent study, \cite{marktest2016}, revels portuguese relationship with \glspl{osn}. This study, has been made by \textit{Marktest Consulting} since 2011, with the goal of know the notoriety, utilization, opinion
and habits of portuguese concerning social networks. The study information was collected trough online interviews. The sample was built from 819 interviews from individuals with age between
15 and 64 years, living in Portugal and using \glspl{osn} in a daily basis.\\
\indent Some of the most interesting facts revealed in this study, relative to the participants are:
\begin{itemize}
  \item 94\% has a Facebook account and 43\% a Youtube account;
  \item 21\% has abandoned a social network in the past year;
  \item 27\% considers that their dedicated time to social media has increased;
  \item 67\% follows celebrities and 62\% follows brands;
  \item 87\% is used to watch videos in social networks.
\end{itemize}

\indent These are indeed interesting conclusions, but what about the top used \glspl{osn}, \textbf{the most used are
the following (by order): Facebook, Youtube, Google+, LinkedIn, Instagram and Twitter}.\\
\indent Relatively to \cite{marktest2016} past studies, Facebook has maintain
its top position, maintaining a grow tendency that has been standing out in the past years.

\indent Going back to Table \ref{table:osns}, we may now comment the usage of \glspl{osn} by portuguese people comparing it
to the global scenario. As one may notice Facebook still rules users preferences within portuguese.
The other noticeable point is that the \glspl{osn} preferred among portuguese are general propose ones,
but with a slight tendency to content sharing networks (mainly photos).

\indent Concerning to global time related usage statistics, according to \cite{marktest2016}, \textbf{portuguese spend 91 minutes a day with social networks},
68\% considers that this is the ideal time to spent with social media, despite 1 in each 4 saying that in the past year has dedicated even more time to them.
Even if people spent more than one hour and an half in this platforms, the study
concluded that \textbf{67\% of the users that visit \glspl{osn} several times a day only 41\% does daily publications}.

\indent \textbf{The prime time for using \glspl{osn} is between 8pm and 10pm}, being the smartphone the most used device in this time. Also in this short period the
featured \gls{osn} is Facebook, the majority of the interviewed say that is the most credible site, the one that provides better and useful information,
the most interesting and addictive.
