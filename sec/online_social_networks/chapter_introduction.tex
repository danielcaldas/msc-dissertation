People need to connect other people, and the urge for connection brings to us what today are known as \glspl{osn}.
These web sites allow us to define a profile as an individual, and to share and visualize content with other individuals in the network, therefore connecting.

\begin{quote}
\textit{"We define Online Social Networks as web-based services that allow individuals to construct a public or semi-public
 profile within a bounded system, articulate a list of other users with whom they share a connection, and view and traverse
 their list of connections and those made by others within the system. The nature and nomenclature of these connections
 may vary from site to site.} \cite{ellison2007social}
\end{quote}

\indent \glspl{osn} have been around for more than a decade now, but these systems have gain world wide popularity since the global adoption of
platforms such as Facebook, Youtube or Twitter, which are platforms that are today massively used across all cultures and age groups, and represents
a paradigm shift on social interaction that we not yet fully understand.\\
\indent The earlier referenced \glspl{osn}, belong to the top of the most visited web sites in the world, that's because these systems not only represents a new
way to keep in touch with friends, but also represents for many, a new way of living, basically we live in network.\\
\indent In this chapter we are going to explore \glspl{osn}, their history, how are these systems being adopted among Internet users, and for some \glspl{osn},
a more detailed and deep study will be conducted for they are important objects of study of this master's thesis.\\
\indent But first, with intent of obtaining a macroscopic perspective of the different \glspl{osn} in the Internet, what they offer
that makes them different from one to another causing many of the users using multiple \glspl{osn} at the same time, we present next a table featuring some
of the most used \glspl{osn}.
