In this chapter we look at our work retrospectively and we discuss the outcoming contribution.\\
\indent At the very start of our work we had limited expectations since the definition of the final product was redefined along the way. We first started studying how \glspl{sn} came to exist and how they were initially perceived (Chapter 2). After realizing the time and effort that sociologists had already invested in this subject, we started investigating the \glspl{osn} that were described as the manifestation of \glspl{sn} of our epoch, the Internet era. The most relevant \glspl{osn} were deeply analyzed in Chapter 3; we looked on how they are composed and what drives users to use them.\\
\indent Then we needed to know how these social structures are studied and interpreted from a scientific perspective; this led us to investigate the work already done in the field of \gls{sna} (Chapter 4). \gls{sna} provides the background to unable us to map social structures in mathematic abstractions. The first step in this analytical process always consists in representing the network by means of a graph. From there on some well established metrics such as \emph{centrality} or \emph{clustering coefficient} shall be evaluated, depending in what we want to perceive. These well established metrics help us to discover a series of facts about a network, such as \emph{how influent are individuals}, \emph{how many communities there are}, is the network dense, as well as all other analysis issues described in Chapter 4.\\
\indent The background provided by that survey on \glspl{sn} and their formal analysis paved the way to design our solution, Socii, in order to define a useful tool that would help users understanding their social structures based on the analysis of the identified \glspl{osn}. In Chapter 5 we discussed our proposal at the most conceptual way discussing its architecture in terms of a block diagram. In Chapter 6 we defined the requirements and their respective priorities in order to obtain a minimum viable product at the end of the project.\\
\indent At the same time that we were developing our tool we documented all the relevant technicalities in Chapter 7. Finally, having Socii been implemented and tested, we reported all the attained results in Chapter 8. This chapter contains a \textit{walkthrough} of the functionalities of our tool; it also includes case studies that demonstrate how the end user can take profit of Socii to obtain concrete results. With this we prove the utility of our final product, with the following main features:
\begin{itemize}
    \item \textbf{Configurable/Parameterized analysis} - we offer our users the possibility to parameterize what metrics they want to calculate upon a given network, this configuration step is transversal to the underlying \glspl{osn} that the user wants to analyze.
    \item \textbf{Clear and intuitive social graph vizualization and interaction} - we built a specialized visual web component that is flexible enough to provide the user a set of visual features such as node coloring, node discovery, node dragging, node labelling etc.
    \item \textbf{Organized overview upon \glspl{sna} and \glspl{osn} data} - we implemented visual components that aggregate both \glspl{sna} metrics and \glspl{osn} information giving the user the opportunity to cross information from both worlds and derive conclusions from intersecting that information. We also integrated features that allow users to compose specific visualization scenarios such as coloring nodes by some common \glspl{osn} property.
\end{itemize}

%% ---------------------------------------------- The main obstacle for Socii
\section{The main obstacle for Socii}
As we have explained along the dissertation, since the beginning we based our work on \glspl{osn}, developing a platform that is data driven meaning that it is built on the assumption of available and accessible data, however in reality this is not happening. Actually the \glspl{osn} we identified and described in Chapter 3 are not \textit{"opening the doors"} to the community providing powerful \glspl{api} in order to make social public data available. That is why we went through the technical and architectural struggle of feeding in Socii networks through a extraction pipeline built on top of web crawlers, that are known and probed to be very slow and error prone. If \glspl{osn} such as Facebook or LinkedIn provided access to their social information via user friendly and opened \glspl{api}, Socii final results could be much more positive and surprising.

%% ---------------------------------------------- Alternative technical approaches that could improve Socii
\section{Alternative technical approaches that could improve Socii}
In this section we will explore alternative approaches that can be implemented in order to improve certain bottlenecks of Socii such as performance. We will list these alternatives explaining both what Socii could gain and loose by selecting those paths.

\subsection{Visualization}

\subsection*{Using WebGL for network visualization and interaction}
Web GL \citep{marrin2011webgl} shall be the best option to build Socii if instead of a two dimensional network representation we choose to go on to the third dimension. This would resolve the node overlapping problem and could make the network discovery task a simpler process, since nodes would have more space to rearrange themselves. Open source projects such as \emph{Graphosaurus} \footnote{A three-dimensional static graph viewer: \url{https://www.npmjs.com/package/graphosaurus}} would be helpful on this implementation, since it offers \textit{"out of the box"} tools for developers to visualize three dimensional graphs.

\subsection{Performance}

\subsection*{Using server side rendering}
Server side rendering is a technique where visual components (templating work) is done in the server side. This normally brings to web applications improvements in terms of the time spent in rendering and building templates, work that is usually done by the client according to architectural definition of more recent front end frameworks and libraries.\\
\indent Server side rendering, in our specific case, could be a good approach since all the heavy calculations for positioning nodes
may be done on the server instead of being done on the browser. This would however have impacts in terms of scalability if we had too many users requesting the rendering of huge networks.

\subsection*{Using web workers for heavy front end background processing}
Modern browsers are close to fully support all the HTML5 new features, this including web workers \citep{webworkers}. These new technologies allow
the browser to run a script operation in background thread separate from the main execution thread of a web application \citep{mdnwebworkers}. For Socii it would be very helpful to have some place where to run calculations as asynchronous tasks. This could be used for example, to metrics calculations instead of the current approach where we need to make an http request to the metrics microservice in order to fetch network metrics.

%% ---------------------------------------------- Socii usage and applications
\section{Socii usage and applications}
We have already described some case studies in Chapter 8 where we demonstrated some of the potential uses of our tool. In this section we will meditate and speculate upon Socii potential of usage across several fields of study. So what could be Socii real applications?

\begin{itemize}
    \item \textbf{Sociology general studies, social analysis} - Basically where Socii is merely a \gls{sna} tool used by scientists and students of the field.
    \item \textbf{Migratory flux of population} - Having a tool such as Socii that allows us to have a macroscopic overview upon social networks we could study population migratory flux (using community detection for example) to understand what is the shape and trends of population migration across the globe. At the time of this writing this could be helpful for example on the detection of refugees communities where we could find what communities were formed with existing and stable communities of other countries and how this affects both refugees and the local population.
    \item \textbf{Society happiness studies/Depression detection among youngsters} - We could use Socii to detect cases of depression among youngsters. This is today unfortunately a very common disease that urges among young people and that could be prevented by monitoring social networks usage among these youngsters and being alter
    for strange/abnormal behaviors.
    \item \textbf{Terrorism awareness/detection} - Using a similar strategy to the one we used to detect refugees communities we could analyze data and look for strange patterns of interactions concerning individuals origins. A simple example could be an individual with nationality X that belongs to a normal network and suddenly starts to create online connections with individuals of the nationality Y. Being this nationality blacklisted as possible association with terrorism, we could see this as a potential threat.
    \item \textbf{Marketing} - As discussed in Chapter 8 one of the use cases was marketing. Actually we could use Socii to detect potential target audiences for a given brand, product or service.
\end{itemize}

%% ---------------------------------------------- Future work
\section{Future work}
In Chapter 6 we already described a lot of improvements that can be done regarding Socii evolution: the implementation of all the requirements that were not marked as \textbf{MUST} requirements are upgrades to the Socii tool that we see as relevant future work. Other ideas to move this project forward are:

\begin{itemize}
    \item Improve network extraction process and allow users to build their networks on the fly;
    \item Adapt the current approach on Socii that builds social structures based on individuals relations to do the same thing with terms/keywords
    building a network of co-related keywords within a restrict domain/theme;
    \item Migrate from using Socii web crawlers to consuming directly these \glspl{api} if eventually \glspl{osn} make their social \glspl{api} available. This would considerably increase user experience and allow us to fulfill the first item of the \textit{future work list} that is to allow users to quickly build their networks.
    \item Understand better Socii positioning among the social analytics world and try to find new and innovative applications for this tool.
\end{itemize}

As an alternative study to the previous list we foreseen, from analyzing various \glspl{osn} in Chapter 3 we have seen a possible research project on creating a framework for building and managing \glspl{osn} automatically and effortlessly. This framework could allow \glspl{osn} to be created on the fly with a model based approach, where the user/programmer just needs to insert a model and the \glspl{osn} would be generated.

% For presentation day think of this things
% - Is this a OAIS system? Study this because they may ask!
% - Who are the people that will be evaluating me and what are their fields of studies?