%% ---------------------------------------------- Chapter Introduction

% REFS
% Alexander Bavelas - Centrality - http://www.analytictech.com/networks/commstruc.htm
Nowadays, it is hard to find something that is not organized as a network, if one tries to understand something about the world around us, then definitely one needs to know something about networks.

Curiously, if we look up the term \gls{sn} in the \cite{dictionary2002cambridge}, we may face the following:

\begin{quote}
\textit{"a website or computer program that allows people to communicate and share information on the Internet using a computer or mobile phone"}
\end{quote}

But, even if today we automatically think in \glspl{sn} as websites (or web applications), deep down we know when talking about \glspl{sn}, we refer to a much more broader term, that said, we may consider a \gls{sn} as the following:

\begin{quote}
\textit{"A social structure made of nodes that are generally individuals or organizations. A social network represents relationships and flows between people, groups, organizations, animals, computers or other information/knowledge processing entities. The term itself was coined in 1954 by J. A. Barnes."}
\cite{webopediasn}
\end{quote}

One may say that networks work like pipes, and through them things flow, from individual to individual inside the network. Trough networks, big institutions can organize themselves, and actually add value to society despite the large number of individuals.


%% ---------------------------------------------- Origins of Social Networks
\section{Origins of Social Networks}

\begin{quote}
\textit{"The network concept is one of the defining paradigms of the modern era."}
\cite{kilduff2003social}
\end{quote}

The network concept is broadly used across multiple fields of study, including, physics, biology, linguistic, anthropology, mathematics, computer science and more recently computer networks.\\

\indent But why is the network approach so adopted in such diversification fields? According to \cite{kilduff2003social}, the answer is because networks allows us to capture the interactions of any individual unit within the larger field of activity to which the unit belongs.

\indent Before reviewing the concept of network (Section 2.2), it is important to talk about it in a sociological perspective.


%% ---------------------------------------------- Sociology Perspective
\section{Sociology Perspective}
\begin{quote}
\textit{"(...) many people attribute the first use of the term ''social network'' to
Barnes (1954). The notion of a network of relations linking social entities, or of webs or ties among social units emanating through society, has
found wide expression throughout the social sciences. (...)"}
\cite{wasserman1994social}
\end{quote}

The \gls{sn} concept has been around for many years now, maybe not in the exact format than nowadays, we are familiarized with the ''\textit{web way}'', in a manner of speaking, but in a more abstract sense, applied in real life within real connections.
\cite{wasserman1994social}, refer that this term has first came into discussion in 1954, introduced by Barnes, J.A.

\begin{quote}
\textit{"Social relations in Bremnes, Norway, fall into three categories: relatively stable formal organizations serving many different
purposes, unstable associations engaged in fishing, and interpersonal links that combine to form a social
network and on which perceptions of class are based. In fishing situations, orders are given and
obeyed; in the other social settings, consensus decisions are reached obliquely and tentatively."}
\cite{barnes1954class}
\end{quote}

In the above citation, John Arundel Barnes, does a very well succeed reflection about the relationships of the people from Bremnes (Norway).\\
\indent The author points out that relations can form organizations for serving a specific purpose, and today we clearly see that the chosen path of
\glspl{sn} and also \glspl{osn}, was narrow down \glspl{sn} to very specific purposes, such as professional networks. So one may say that John
Arundel Barnes not only coined the term \gls{sn}, but also was one of the first who described \textbf{interest-based social networks}.\\


%% ---------------------------------------------- Fundamental Concepts
\section{Fundamental Concepts}

The concepts listed below are of key importance and are the basis of comprehension of \glspl{sn} (\cite{wasserman1994social}).

\begin{itemize}
    \item \emph{Actor} - It is important to understand the linkages among social entities and the implications of these linkages, these social entities are described as actors. Actors are are discrete individual, corporate, or collective social units.
    \item \emph{Relational Tie} - Actors are linked to one another trough \textit{social ties}. The type of ties may be extensive, and it describes the nature of the connection. Some example of ties:
        \begin{itemize}
            \item \textbf{Evaluation} of one person by another;
            \item \textbf{Transference} of resources (business transactions);
            \item \textbf{Association} (to social event or cause);
            \item \textbf{Behavioural} interactions (communicating);
            \item \textbf{Moving} between places or statuses (migration, social or physical mobility);
            \item Others may be: physical connection (roads, rivers), formal relations (authority), biological relationship.
        \end{itemize}
    \item \emph{Dyad} - The most basic relationship that can be established is a dyad, a connection between two actors.
    \item \emph{Triad} - A relation established between three actors. Many studies included breaking \glspl{sn} down to small groups (triads), this allowed a more clear conclusion about the transitivity of the connections.
    \item \emph{Subgroup} - It defines any subset of actors in a \gls{sn} (conceptually, subgroups come after dyads and triads).
    \item \emph{Group} - A finite set of actors who for conceptual, theoretical or empirical reasons are treated as a finite set of individuals in which network measurements are made.
    \item \emph{Relation} - A collection of ties of a specific kind among members of a group is called a \textbf{relation} (e.g. a connection in \textit{LinkedIn} is a relation while evaluating our connections of sending them messages are ties).
    \item \emph{\gls{sn}} - At last, with the definitions of actor, group and relation, a \gls{sn} consists of a finite set or sets of actors and the relation or relations defined on them. The presence of relation information is critical and defining feature of a \gls{sn}.
\end{itemize}

Next, we present a two more advanced and abstract concepts but still fundamental concerning \glspl{sn} in the context of this project.

\subsection*{Homophily}

In a New York Times Magazine article (\cite{nytmagazinehomop}) it is mentioned that the term \textit{"homophily"}, was coined in the 1950s by sociologists and in a more literal sense it means \textit{"love the same"}. This term emerges from the natural tendency we have to link to other individuals that are similar to us.\\
\indent Quoting the sociologists \cite{mcpherson2001birds}, \textbf{\textit{“Similarity breeds connection”}}, basically similarity is considered a generator of connections among individuals, being the result of this phenomena homogeneous \glspl{sn}.\\
\indent The term \textit{homophily} has been cited in light of many different themes, from teenagers choosing friends who drink and smoke similar amounts to theirs, or in explaining how homophily influences the matches of partners in online social dating, this proving that one likes, most of the time, someone like oneself, on or offline (\cite{fiore2005homophily}).\\
\indent From another point of view, this trend could be seen as a threat to diversity and globalization. It is said that diversity can be a synonym of power, when bringing different cultures and different ways of thinking together we could achieve great things, but homophily is already a cemented concept/pattern that sociologists observe among \glspl{sn}, and maybe we could find ways to battle in favor of diversity, or maybe homophily is a fundamental property in order to structure society.

\subsection*{Heterophily}

In order to complete the previous presented concept (\textit{homophily}), we now present the opposite that is \textit{heterophily}, that translates in literally the opposite idea, being \textit{heterophily} the trend of individuals belonging to diverse groups thus connecting with different people.


%% ---------------------------------------------- Abstraction and Generalization
\section{Abstraction and Generalization}

In a more abstract sense networks are merely abstractions that are originated by the generalization both of individuals, and relationships.\\

\begin{quote}
\textit{"When we study social organization of a simple society, we aim at comprehending all the various ways in which the members os the society systematically interact with one another. For purposes of analysis we treat the political system, the pattern of village life, the system of kinship and affinity, and other similar areas of interaction as parts of the same universe of discourse, as tough they were of equal analytical status, and we strive to show how the same external factors, principals of organization and common values influence these different divisions of social life.
"}
\cite{barnes1954class}
\end{quote}

\indent In the above citation, the author describes a generalist approach on analyzing social networks. The two main characteristics of this approach are \textbf{generalization} and \textbf{abstraction}. First generalization because we are trying to simplify reality by minifying different kinds of connections (political, affinity etc.), this will allow us to treat networks as part of a world where they can fit in the domain of the exact sciences, being mathematical the way networks express themselves in order to measure metrics and behavior analysis.\\
\indent Abstraction comes naturally in the way as the process of generalization takes laces, we could see abstraction and generalization as synonym in this specific case, but it also may be seen as a tool to see through the generalization process. Also fitting (at least try) networks and their analysis within the domain of exact sciences, requires the abstraction of the generalization that toke place before. In Chapter 4 we will cover with much more detail the field known as \glspl{sna}, that is responsible of deriving conclusions from analyzing social structures.
