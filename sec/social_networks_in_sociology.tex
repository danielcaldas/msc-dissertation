%% ---------------------------------------------- Chapter Introduction

\indent \indent Nowadays is hard to find something that is not organized as a network, if one tries to understand something about the world around us, then definitely one needs to know something about networks.

Curiously if you look up the term ''\textit{Social Network}'' in the Cambridge Dictionary, we may face the following:

\hfill \break

\hypertarget{cambridge_dict_sn_org}{}
\say{a website or computer program that allows people to communicate and share information on the Internet using a computer or mobile phone}
%\cite{cambridge_dict_sn}

\hfill \break

But, even if today we automatically think in SNs as websites (or web applications), deep down we know when talking about SNs, we refer to a much more broader term, that said, we may consider a SN as the following:

\hfill \break

\hypertarget{webopedia_sn_defenition_org}{}
\say{A social structure made of nodes that are generally individuals or organizations. A social network represents relationships and flows between people, groups, organizations, animals, computers or other information/knowledge processing entities. The term itself was coined in 1954 by J. A. Barnes.}
%\cite{webopedia_sn_defenition}

\hfill \break

One may say that networks work like pipes, and trough them things flow, from individual to individual inside the network. It's trough networks that big institutions can organise themselves, and actually add value to society despite the large number of individuals.

%% ---------------------------------------------- Origins of Social Networks
\section{Origins of Social Networks}

\hypertarget{sna_bible_org}{}
\say{(...) many people attribute the first use of the term "social network " to
Barnes (1954). The notion of a network of relations linking social entities, or of webs or ties among social units emanating through society, has
found wide expression throughout the social sciences. (...)}
%\cite{sna_bible}

\hfill \break

The Social Network concept has been around for many years now, maybe not in the exact format that nowadays, we are familiarized with (''\textit{web way}'', in a manner of speaking), but in a more abstract sense, applied in real life within real connections.
In "\textit{Social Network Analysis - Methods and Applications Stanley Wasserman and Katherine Faust"}, the authors refer that this term has first came into discussion in 1954, introduced by Barnes, J.A.

\hfill \break

\hypertarget{barnes_norwegian_org}{}
\say{Social relations in Bremnes, Norway, fall into three categories: relatively stable formal organizations serving many different
purposes, unstable associations engaged in fishing, and interpersonal links that combine to form a social
network and on which perceptions of class are based. In fishing situations, orders are given and
obeyed; in the other social settings, consensus decisions are reached obliquely and tentatively.}
%\cite{barnes_norwegian}