%% ---------------------------------------------- Chapter Introduction

\indent \indent Nowadays is hard to find something that is not organized as a network, if one tries to understand something about the world around us, then definitely one needs to know something about networks.

Curiously if you look up the term \textit{''social network''} in the Cambridge Dictionary, we may face the following:

\begin{quote}
\textit{"a website or computer program that allows people to communicate and share information on the Internet using a computer or mobile phone"}
\end{quote}
%\cite{cambridge_dict_sn}

But, even if today we automatically think in SNs as websites (or web applications), deep down we know when talking about SNs, we refer to a much more broader term, that said, we may consider a SN as the following:

\begin{quote}
\textit{"A social structure made of nodes that are generally individuals or organizations. A social network represents relationships and flows between people, groups, organizations, animals, computers or other information/knowledge processing entities. The term itself was coined in 1954 by J. A. Barnes."}
\end{quote}
%\cite{webopedia_sn_defenition}

One may say that networks work like pipes, and trough them things flow, from individual to individual inside the network. It's trough networks that big institutions can organize themselves, and actually add value to society despite the large number of individuals.

%% ---------------------------------------------- Origins of Social Networks
\section{Origins of Social Networks}

\begin{quote}
\textit{''The network concept is one of the defining paradigms of the modern era.''} [REF\_BOOK]
\end{quote}

\indent Before talking of network from the sociology perspective, one needs to review the network concept, which is broadly used across multiple fields of study, this include, physics, biology, linguistic, anthropology, mathematics, computer science and more recently computer networks.

\indent But why is the network approach so adopted in such diversification fields? The answer is, because networks allows us to capture the interactions of any individual unit within the larger field of activity to which the unit belongs [REF\_BOOK].


\subsection{Sociology Perspective}
\begin{quote}
\textit{"(...) many people attribute the first use of the term ''social network'' to
Barnes (1954). The notion of a network of relations linking social entities, or of webs or ties among social units emanating through society, has
found wide expression throughout the social sciences. (...)"}
\end{quote}
%\cite{sna_bible}

The \textit{''social network''} concept has been around for many years now, maybe not in the exact format that nowadays, we are familiarized with (''\textit{web way}'', in a manner of speaking), but in a more abstract sense, applied in real life within real connections.
In \textit{"Social Network Analysis - Methods and Applications Stanley Wasserman and Katherine Faust"}, the authors refer that this term has first came into discussion in 1954, introduced by Barnes, J.A.

\begin{quote}
\textit{"Social relations in Bremnes, Norway, fall into three categories: relatively stable formal organizations serving many different
purposes, unstable associations engaged in fishing, and interpersonal links that combine to form a social
network and on which perceptions of class are based. In fishing situations, orders are given and
obeyed; in the other social settings, consensus decisions are reached obliquely and tentatively."}
\end{quote}
%\cite{barnes_norwegian}

In the above citation, John Arundel Barnes, does a very well succeed reflection about the relationships of the people from Bremnes (Norway). The author points out that relations can form organizations for serving a specific purpose, and today we clearly see that the chosen path of SNs and also Online Social Networks (OSNs), was narrow down social networks to very specific purposes, such as professional networks. So one may say that John Arundel Barnes not only coined the term \textit{''social network''}, but also was one of the first who described \textbf{interest-based social networks}.

\section{Relevant SN related terms}
\textbf{In this section talk about some inherent concepts of social networks, \underline{only if they are found relevant}.}
(Review this theories. Why are they important in sociology? What is their placement (fitting) in the thesis?)
\begin{itemize}
\item Homophily and Heterophily
\item Structuralism
\item Structural functionalism
\item Conflict theories
\item Social constructionism
\end{itemize}