\begin{table}[H]
\hspace*{-1.25in}
\renewcommand{\tabcolsep}{2pt}
\begin{tabular}{ |c|c|c|c|l|  }
\hline
\textbf{Name} & \textbf{Year of launch} & \textbf{Registered Users} & \textbf{Active Users} & \textbf{Description/Purpose}\\
\hline
\cellcolor{gray!60}Facebook & 2004 & \textgreater 1 712 000 000 & 1 712 000 000 & \textbf{General}. Photos, videos, blogs, apps.\\
\hline
\cellcolor{gray!60}Google+ & 2011 & 1 600 000 000 & 300 000 000 & \begin{tabular}{@{}l@{}}\textbf{General}. Google+ is an interest-based\\social network that is owned\\and operated by Google.\end{tabular}\\
\hline
\cellcolor{gray!60}Youtube & 2005 & \textgreater 1 000 000 000 & 1 000 000 000 & \begin{tabular}{@{}l@{}}Allows billions of people to discover,\\watch and share originally-created videos.\\Provides a forum for people to connect,\\ inform, and inspire others.\end{tabular}\\
\hline
\cellcolor{gray!30}Qzone & 2005 & \textgreater 652 000 000 & 652 000 000 & \begin{tabular}{@{}l@{}}\textbf{General}. It allows users to write blogs,\\keep diaries, send photos, listen to music,\\and watch videos.\\It's only available in Chinese.\end{tabular}\\
\hline
\cellcolor{gray!30}Twitter & 2006 & 645 750 000 & 313 000 000 & \textbf{General}. Micro-blogging, RSS, updates.\\
\hline
\cellcolor{gray!30}Tumblr & 2007 & \textgreater 555 000 000 & 555 000 000 & \begin{tabular}{@{}l@{}}Microblogging platform and social networking website.\end{tabular}\\
\hline
\cellcolor{gray!30}Instagram & 2010 & \textgreater 500 000 000 & 500 000 000 & A photo and video sharing site.\\
\hline
\cellcolor{gray!30}LinkedIn & 2003 & \textgreater 450 000 000 & 106 000 000 & Business and professional networking.\\
\hline
\cellcolor{gray!30}Sina Weibo & 2009 & 300 000 000 & 282 000 000 & \begin{tabular}{@{}l@{}}Social microblogging site in mainland China.\end{tabular}\\
\hline
\cellcolor{gray!30}VK & 2006 & 249 409 900 & 100 000 000 & \begin{tabular}{@{}l@{}}\textbf{General}, including music upload, listening and search.\\Popular in Russia and former Soviet republics.\end{tabular} \\
\hline
\cellcolor{gray!30}Reddit & 2005 & 234 000 000 & 120 000 000 & \begin{tabular}{@{}l@{}}Social media, social news aggregation, web\\content rating, and discussion website.\end{tabular}\\
\hline
\cellcolor{gray!30}Vine & \textbf{2013} & 200 000 000 & 100 000 000 & \begin{tabular}{@{}l@{}}Short-form video sharing service where\\ users can share six-second-long looping video clips.\end{tabular}\\
\hline
\cellcolor{gray!30}Pinterest & 2010 & 176 000 000 & 100 000 000 & \begin{tabular}{@{}l@{}}The world’s catalog of ideas. Find and save\\recipes, parenting hacks, style inspiration and\\other ideas to try.\end{tabular}\\
\hline
\cellcolor{gray!30}Flickr & 2007 & 112 000 000 & 92 000 000 & \begin{tabular}{@{}l@{}}Helping people make their photos\\ available to the people who matter to them.\\Enable new ways of organizing\\photos and video.\end{tabular}\\
\hline
\cellcolor{gray!10}Meetup & \textbf{2002} & 27 590 000 & - & \begin{tabular}{@{}l@{}}World's largest network of local groups.\\Meetup makes it easy for anyone\\to organize a local group or find\\one of the thousands already meeting\\up face-to-face. \cite{meetup}\end{tabular}\\
\hline
\cellcolor{gray!10}Couchsurfing & 2004 & 12 000 000 & - & \begin{tabular}{@{}l@{}}Couchsurfing connects travelers with\\a global network of people willing\\to share in profound and meaningful ways,\\ making travel a truly\\ social experience. Is commonly used by travelers\\to find free hosts across the globe.\\\cite{csurf}\end{tabular}\\
\hline
\cellcolor{gray!10}ResearchGate & 2008 & \textgreater 11 000 000 & - & \begin{tabular}{@{}l@{}}Built by scientists, for scientists.\\Connect the world of\\ science and make\\ research open to all. \cite{rgate}\end{tabular}\\
\hline
\end{tabular}
\caption{\label{table:osns} Table describing most used \glspl{osn}. (\cite{statista}, \cite{expandedramblings})}
\end{table}

\indent Table \ref{table:osns} lists the most used and popular \glspl{osn}, \textbf{ordered by the estimated number of registered users}.
Also notice that, for those \gls{osn} where the number of registered users is unknown, we will assume that it is a larger value than
the monthly active users represented by the column \textit{Active Users}.
\\
\indent The first obvious comment on the listed \glspl{osn} is that general purpose \glspl{osn} have more users (social
networks with the word \textit{General} in bold), being Youtube an exception, since it is not a general purpose \glspl{osn}, neither
is focused on individuals, it is build around \textbf{social objects}, the videos.
\\
\indent The grey scale in the first column of Table \ref{table:osns} divides \glspl{osn} in three groups: the first and smallest, the 1 billion
or more users \glspl{osn}; the second the \glspl{osn} with less than 1 billion users and more then 100 million; finally, the third group, \glspl{osn} with
less then 100 million users. At this point, we begin to observe that \textbf{the narrower purpose \glspl{osn}} such as ResearchGate (mainly for researchers) or
Couchsurfing (mainly for open minded travelers), \textbf{have a smaller number of registered users}, which is expected since the target audience is also smaller.
\\
\indent Other \glspl{osn} not listed in Table \ref{table:osns}, but still worth mentioning include \textbf{Classmates} (helps users finding
classmates form kindergarten, primary school, high school etc.) known for being one of the first \glspl{osn}, since it was
launched in 1995, and \textbf{Ask.fm} (allows users to interact with other users asking and answering questions (revealing identity is optional)).
\\
\indent An important note on the listed \glspl{osn} in Table \ref{table:osns} is that only Qzone, Vine, Couchsurfing and ResearchGate don't provide any web APIs
to fetch data or publish content, while all the others offer a wide variety of web services for developers to consume and use as they please, of course within the terms and policies
of use of each \gls{osn}.
