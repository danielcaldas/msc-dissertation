\section{How Online Social Networks Have Changed The World}

Social media have clearly shifted the way we communicate and we perceived the world, simply putting it, nowadays with social media one can say that social media is responsible for \textit{"everyone talking to everyone about everything all of the time"}.\\
\indent According to \cite{demogsmu}, 62\% of the entire adult population in on social media. As an example of events that were clearly influenced by social media, we have the presidential campaign of Barack Obama in the United States, started in 2007 and ended in 2009, Barack Obama had as his campaign technological adviser Chris Hughes, co-funder of Facebook, who played a crucial role in the camping trough online social media. The outcome of the election of 2009 could have been very different without the online social media.\\
\indent According to \cite{6wayschange}, a very interesting reflection is made on how social media impact the world, and the six major drawn conclusions are the following: \textbf{across industries, social media is going from a \textit{“nice to have”} to an essential component of any business strategy}; \textbf{social media platforms may be the banks of the future}, as example we have the bank customer profiling trough social media in order to get a loan; \textbf{social media is shaking up healthcare and public health}, because information is spreader \textit{at the speed of light} trough social media, this means less struggle to achieve public health and well being awareness; \textbf{social media is changing how we govern and are governed}, with \glspl{osn} public participation has grown and everyone can participate in their opinion making people voices louder, bringing more credibility to the democratically system implemented by many governments across the planet; \textbf{social media is helping us better respond to disasters}, as the health public awareness improved trough social media information propagation speed, so did improved the response of governments and institutions to disasters such as natural disasters, in countries that may have not the services or infrastructure to respond to some catastrophes, making social media and crucial component to raise awareness across the globe, that have impact in help mobility, or fund raising for supports the damages made by certain disaster; \textbf{social media is helping us tackle some of the world’s biggest challenges, from human rights violations to climate change}.\\
\indent If we look particularly to the most globally used \gls{osn}, as reported by \cite{relrevfacebook}, there are pointed out \textit{"seven ways Facebook has changed the world"}, we are going to point and comment out some of the more relevant. \textbf{Facebook has changed the definition of friend}, if back there having a dozen of friends was already a very large number of relationships, with Facebook the new limit was raised up to the hundreds or thousands of friends, the concept was given a completely new meaning, since we don't need to know a person face to face so that one becomes friend with the other, one simply needs to click the \textit{"add friend"} button, and it does not matter if it is one's neighbor or some other person on the another side of the planet; \textbf{We care less about privacy}, \textit{"if you are not paying for it, you are the product"}, means that we are not paying for using Facebook or any other \glspl{osn}, this said we must retain that these online platform profit from our information and from our interactions, but even being the major of the users aware of this situation, that doesn't seems to bother anyone; \textbf{Facebook has created millions of jobs – but not in its own offices}, for example the marketing industry suffer a revolution since the raise of the social media, there are jobs for people to manage business and brands profiles on \glspl{osn} it's also a new way to approach customers, as we have seen previously with banks; \textbf{Facebook has been the tool to organize revolutions}, protests and awareness campaigns are raised inside facebook, this is related to the political influence and awareness capacity that we previously have pointed out in this same section.\\
\indent Now switching to the negative aspects of not only Facebook but \glspl{osn} and social media in general. Very strong campaigns were raised against social media, for instance, \href{http://www.imdb.com/title/tt3333168/}{\textit{"The Anti-Social Network"}} a short film depicting a life of an adult which as become obsessed with social networking at the point he starts to break boundaries between his real life and his virtual one. Strategically or ironically this campaigns use social media to spread the word.\\
\indent We have seen that social media had a great deal of impact in society, what about our bodies?  There are numeral studies on this matter, focusing on finding the true negative impacts of \glspl{osn} on
our personal health. According to \cite{lin2012abnormal}, scans to brains of people how excessively use social media, point out that there is a clear degradation of white matter similar to people who are addicted to substances such as drugs or alcohol, in the regions that control emotional processing, attention and decision making, because social media immediate reward (instant feedback) with very small effort, this causes the brain rewire itself make us to desire this stimulations \cite{berridge1998role}. Another common situation among \glspl{osn} users is the idea of multitasking, the felling that one is able to being productive in some task while browsing on social media. Well according \cite{ophir2009cognitive} users who heavily use social media are more susceptible to interference from irrelevant environmental stimuli, leading this users to perform worse on a test of task-switching ability, because they were not able to filter out interferences.
