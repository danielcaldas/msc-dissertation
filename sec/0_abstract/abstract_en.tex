% RPD
% This document represents the work developed under the master's thesis Analysis of Visualization of Social Networks until the present date.
% First we present and identify the problem and challenge of building a system for network analysis, we also mention the motivation,
% research hypothesis and goals. From here we start to search the science foundations for social networks, starting from the very basic theoretical
% concepts in Chapter two, then we travel to the present and present Online Social Networks as the most known application of this science, in Chapter 3
% we do a more detailed study on Online Social Networks starting from exposing theme in a more generalist way and then narrowing and exploring some of then with more detail. In Chapter 4 we cover the analysis theoretical with the tool we want to build in mind, being Social Network Analysis a very broad field we take advantage of our goals to narrow the research on this field, presenting only the concepts and tools that in a certain way built the path for Chapter 5, where we propose an yet non much detailed architecture schema and description of some features that we think that our system should implement.

\begin{quote}
\textit{\textbf{"You can think of networks as vast fabrics of humanity, and we all occupy particular spots within the network."} Nicholas Christakis}
\end{quote}

This document represents the study developed under the master's thesis Analysis of Visualization of Social Networks, that overlaps
two main scientific fields, sociology (more concisely social networks) and computer science, aiming at the design and implementation of a system for social network analysis.\\

Nowadays we face an age of massive Internet usage, with Online Social Networks we practically live this parallel reality where everything we do and everyone we met is exposed and shared through these online \textit{''worlds''}. Today, being able to study and understand how information flows and how relationships are built within these online networks is of paramount importance for various reasons, these can be social, educational, political or economical. This master work studied sociology, social network analysis, and computer science to employ the researched material aiming at building a tool that allows users to explore their social structure in order to derive sophisticated conclusions, that wouldn't normally come up when they are browsing through their online feeds, because we provide to the end user a personalized, macroscopic and objective perspective of their social network.