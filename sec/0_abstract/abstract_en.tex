% RPD
% This document represents the work developed under the master's thesis Analysis of Visualization of Social Networks until the present date.
% First we present and identify the problem and challenge of building a system for network analysis, we also mention the motivation,
% research hypothesis and goals. From here we start to search the science foundations for social networks, starting from the very basic theoretical
% concepts in chapter two, then we travel to the present and present Online Social Networks as the most known application of this science, in chapter 3
% we do a more detailed study on Online Social Networks starting from exposing theme in a more generalist way and then narrowing and exploring some of then with more detail. In chapter 4 we cover the analysis theoretical with the tool we want to build in mind, being Social Network Analysis a very broad field we take advantage of our goals to narrow the research on this field, presenting only the concepts and tools that in a certain way built the path for chapter 5, where we propose an yet non much detailed architecture schema and description of some features that we think that our system should implement.

\begin{quote}
\textit{\textbf{"You can think of networks as vast fabrics of humanity, and we all occupy particular spots within the network."} Nicholas Christakis}
\end{quote}

This document represents the study developed under the master's thesis Analysis of Visualization of Social Networks, that overlaps
two main scientific fields, sociology (more concisely social networks) and computer science, aiming at the design and implementation of a system for social network analysis.\\

\indent First we present and identify the problem and challenge of building a system for social network analysis, we also mention the motivation, research hypothesis and goals. From here we start to search the scientific foundations for social networks, starting from the very basic theoretical concepts in chapter two, traveling to the present to introduce Online Social Networks as the most well known application of this science. In chapter 3 we go deep into details about Online Social Networks starting from exposing them with a more general discussion of this topic around the world and in the context of the Portuguese society, and exploring deeper seven of the most popular networks available on the Web at present. In chapter 4 we provide the theoretical background regarding the scientific analysis of Social Networks, having always in mind with the tool we want to build; being Social Network Analysis a very broad field we decided to focus the research on this field on the master's thesis objectives, presenting only the concepts and tools that in a certain way built the path for chapter 5, where we propose the system architecture.\\

\indent Chapters 6 and 7 are more technical, we discuss and prioritize the requirements for our system and we present our technological choices based on a proof of concept built previously to the system implementation only for architectural and technological study of viability. In chapter 8, using a case study, we present the developed tool, the final results and respective analysis. At the end we discuss the main contributions of this work, and where could we go from here; to close chapter 9, we present different possible directions for future research work.
