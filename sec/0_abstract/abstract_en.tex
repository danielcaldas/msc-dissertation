This document represents the work developed under the master's thesis Analysis of Visualization of Social Networks until the present date.
First we present and identify the problem and challenge of building a system for network analysis, we also mention the motivation,
research hypothesis and goals. From here we start to search the science foundations for social networks, starting from the very basic theoretical
concepts in chapter two, then we travel to the present and present Online Social Networks as the most known application of this science, in chapter 3
we do a more detailed study on Online Social Networks starting from exposing theme in a more generalist way and then narrowing and exploring some of then with more detail. In chapter 4 we cover the analysis theoretical with the tool we want to build in mind, being Social Network Analysis a very broad field we take advantage of our goals to narrow the research on this field, presenting only the concepts and tools that in a certain way built the path for Chapter 5, where we propose an yet non much detailed architecture schema and description of some features that we think that our system should implement.
