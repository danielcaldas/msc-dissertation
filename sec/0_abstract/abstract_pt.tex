% RPD
% O presente relatório de pré-dissertação, representa o tabalho desenvolvido sobre o tema de Análise e Visualização de Redes Sociais Dinâmicas, até à presente data.
% Começamos por apresentar e identificar o problema e desafio de construír um sistema para análise de redes sociais, mencionando os nossos objetivos e motivação para o
% projeto. No segundo capítulo começamos a explorar as fundações científicas sobre redes sociais, apontando os conceitos mais básicos. No terceiro capítulo fazemos uma passagem pelas Redes Sociais Online, começando por fazer um levantamento mais generalista acerca do tema e das redes sociais online, após esta introdução, fazemos um estudo mais detalhado acerca de algumas redes sociais que selecionamos. No seguinte capítulo apresentamos a base teórica  da Análise de Redes Sociais, uma área bastante basta e complexa. Por ser uma área de grande dimensão e diversas aplicações, limitamos o seu estudo ao propósito desta dissertação, sendo que o critério para exploração de alguns conceitos terão como base o facto de nos ser útil a sua compreensão para a implementação do sistema, cuja arquitetura ainda numa fase embrionária, é apresentada no capítulo final.

\begin{quote}
\textit{\textbf{"Podemos ver as redes sociais como vastas fábricas de humanidade, onde cada um de nós ocupa um lugar específico."} Nicholas Christakis}
\end{quote}

O presente documento representa o estudo desenvolvido sob a dissertação de mestrado de Análise e Visualização de Redes Sociais Dinâmicas, que resulta sobretudo da intersecção de dois ramos científicos, a sociologia e as ciências da computação, com o objectivo de propor o desenho e implementação de um sistema de análise de redes sociais.\\

\indent Começamos por apresentar e identificar o problema e desafio de construír um sistema para análise de redes sociais, mencionando os nossos objetivos e motivação para o projeto. No segundo capítulo começamos a explorar as fundações científicas sobre redes sociais, apontando os conceitos mais básicos. No terceiro capítulo fazemos uma passagem pelas Redes Sociais Online, começando por fazer um levantamento mais generalista acerca do tema e das redes sociais online, após esta introdução, fazemos um estudo mais detalhado acerca de algumas redes sociais que selecionamos. No seguinte capítulo apresentamos a base teórica  da Análise de Redes Sociais, uma área bastante basta e complexa. Por ser uma área de grande dimensão e diversas aplicações, limitamos o seu estudo ao propósito desta dissertação, sendo que o critério para exploração de alguns conceitos terão como base o facto de nos ser útil a sua compreensão para a implementação do sistema, cuja arquitetura apresentamos no capítulo seguinte.\\

\indent Os próximos passos têm um teor mais técnico, discutimos e prioritizamos os requisitos do nosso sistema, apresentamos as tecnologias com base na contrução de uma pequena prova de conceito que também é apresentada nestes capítulos para fins de validação da viabilidade da arquitectura e das tecnologias escolhidas. Numa fase mais final apresentamos a ferramenta construída e os resultados obtidos, para terminar com uma discussão acerca das conclusões mais relevantes a retirar deste projecto, bem como algumas possibilidades de trabalho futuro.
