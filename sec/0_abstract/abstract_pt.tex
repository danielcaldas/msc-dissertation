% RPD
% O presente relatório de pré-dissertação, representa o tabalho desenvolvido sobre o tema de Análise e Visualização de Redes Sociais Dinâmicas, até à presente data.
% Começamos por apresentar e identificar o problema e desafio de construír um sistema para análise de redes sociais, mencionando os nossos objetivos e motivação para o
% projeto. No segundo capítulo começamos a explorar as fundações científicas sobre redes sociais, apontando os conceitos mais básicos. No terceiro capítulo fazemos uma passagem pelas Redes Sociais Online, começando por fazer um levantamento mais generalista acerca do tema e das redes sociais online, após esta introdução, fazemos um estudo mais detalhado acerca de algumas redes sociais que selecionamos. No seguinte capítulo apresentamos a base teórica  da Análise de Redes Sociais, uma área bastante basta e complexa. Por ser uma área de grande dimensão e diversas aplicações, limitamos o seu estudo ao propósito desta dissertação, sendo que o critério para exploração de alguns conceitos terão como base o facto de nos ser útil a sua compreensão para a implementação do sistema, cuja arquitetura ainda numa fase embrionária, é apresentada no capítulo final.

\begin{quote}
\textit{\textbf{"Podemos ver as redes sociais como vastas fábricas de humanidade, onde cada um de nós ocupa um lugar específico."} Nicholas Christakis}
\end{quote}

O presente documento relata o estudo desenvolvido no âmbito do trabalho de mestrado do autor
sobre Análise e Visualização de Redes Sociais Dinâmicas conducente à tese que aqui se expõe e defende, trabalho esse que resulta sobretudo da intersecção de dois ramos científicos, a sociologia e as ciências da computação, com o objectivo de propor o desenho e implementação de um sistema de análise de redes sociais.\\

Vivemos atualmente numa era de uso massivo da Internet. Com as Redes Sociais Online, acessíveis através da Internet,
experienciamos uma espécie de realidade paralela onde todas as pessoas com quem convivemos e praticamente tudo o que fazemos é exposto e partilhado através destes \textit{''mundos''} virtuais. Na atualidade, a capacidade de estudar e compreender de que forma a informação flui e como se constroem novos relacionamentos dentro destas redes online é de extrema importância por diversos fatores, podendo estes ser de ordem social, educativa, política ou económica. No âmbito desta dissertação de mestrado estudamos sociologia, análise de redes sociais e ciências da computação com o objectivo de construir uma ferramenta que permita aos utilizadores explorar as suas estruturas sociais para que possam chegar a conclusões mais sofisticadas, conclusões que não surgiriam simplesmente por navegarem num perfil duma rede social online. Com a nossa ferramente providenciamos ao utilizador final uma perspectiva personalizada, macroscópica e objectiva da sua rede social.