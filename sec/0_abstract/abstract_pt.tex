% RPD
% O presente relatório de pré-dissertação, representa o tabalho desenvolvido sobre o tema de Análise e Visualização de Redes Sociais Dinâmicas, até à presente data.
% Começamos por apresentar e identificar o problema e desafio de construír um sistema para análise de redes sociais, mencionando os nossos objetivos e motivação para o
% projeto. No segundo capítulo começamos a explorar as fundações científicas sobre redes sociais, apontando os conceitos mais básicos. No terceiro capítulo fazemos uma passagem pelas Redes Sociais Online, começando por fazer um levantamento mais generalista acerca do tema e das redes sociais online, após esta introdução, fazemos um estudo mais detalhado acerca de algumas redes sociais que selecionamos. No seguinte capítulo apresentamos a base teórica  da Análise de Redes Sociais, uma área bastante basta e complexa. Por ser uma área de grande dimensão e diversas aplicações, limitamos o seu estudo ao propósito desta dissertação, sendo que o critério para exploração de alguns conceitos terão como base o facto de nos ser útil a sua compreensão para a implementação do sistema, cuja arquitetura ainda numa fase embrionária, é apresentada no capítulo final.

\begin{quote}
\textit{\textbf{"Podemos ver as redes sociais como vastas fábricas de humanidade, onde cada um de nós ocupa um lugar específico."} Nicholas Christakis}
\end{quote}

O presente documento representa o estudo desenvolvido no âmbito dissertação de mestrado sobre Análise e Visualização de Redes Sociais Dinâmicas conducente a esta dissertação, trabalho esse que resulta sobretudo da intersecção de dois ramos científicos, a sociologia e as ciências da computação, com o objectivo de propor o desenho e implementação de um sistema de análise de redes sociais.\\

\indent Começamos por apresentar e identificar o problema e desafio de construír um sistema para análise de redes sociais, mencionando os nossos objetivos e motivação para o projeto. No segundo capítulo começamos a explorar as fundações científicas sobre redes sociais, apontando os conceitos mais básicos. No terceiro capítulo fazemos uma passagem pelas Redes Sociais Online, começando por fazer um levantamento mais generalista acerca do tema à volta do mundo e em particular no âmbito da sociedade portuguesa; após esta introdução, fazemos um estudo mais detalhado sobre sete redes sociais, atualmente acessíveis na Web, que selecionamos por julgarmos serem as mais usadas e expressivas deste gigantesco fenômeno social. No capítulo 4 apresentamos a base teórica  da Análise de Redes Sociais, uma área bastante basta e complexa. Por ser uma área de grande dimensão e para a qual existem já disponíveis diversas aplicações, limitamos o seu estudo ao propósito deste trabalho de mestrado, sendo que o critério para exploração de alguns conceitos vê como base o facto de nos ser útil a sua compreensão para a implementação do sistema, cuja arquitetura apresentamos no capítulo 5.\\

\indent Os próximos capítulos 6 e 7 são de um teor mais técnico; neles discutem-se e organizam-se por prioridades os requisitos do nosso sistema e decidem-se as tecnologias a usar, tendo a escolha sido feita com base na construção de uma pequena prova de conceito, também apresentada nestes capítulos, desenvolvida precisamente para validar a viabilidade da arquitectura e comparar abordagens de trabalho e tecnologias. Quase a fechar, no capítulo 8, apresentamos a ferramenta construída e os resultados obtidos, discutindo-se ainda um caso de estudo para ilustrar o processo e a ferramenta Socii. Para terminar faz-se, no capítulo 9, uma discussão acerca das conclusões mais relevantes a retirar deste projecto e seus contributos, bem como se discutem algumas possibilidades de trabalho futuro.
