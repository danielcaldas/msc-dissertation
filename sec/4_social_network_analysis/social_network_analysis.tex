\acrfull{sna} is the study of how people are connected to each other, basically it studies a set of relations among a set of entities,
these entities may be individuals, organizations, or even countries.\\\\
\indent The common analysis procedure consists in mapping the network and then creating metrics to
characterize the network. Then one tries to figure what is the structure of the network and why does
it have that structure. \glspl{sna} is also about looking at the individuals inside the network and where are those individuals located.

%% ---------------------------------------------- Fundamental Concepts for Network Analysis
\section{Fundamental Concepts for Network Analysis}

According to \cite{wasserman1994social}, the concepts listed below are of key importance to understand \glspl{sna}.

\begin{itemize}
    \item \emph{Actor} - \gls{sna} is concerned with understanding the linkages among social entities and the implications of these linkages, these social entities are described as actors. Actors are  discrete individual, corporate, or collective social units.
    \item \emph{Relational Tie} - Actors are linked to one another through \textit{social ties}. The type of ties may be extensive, and it describes the nature of the connection. Some example of ties:
        \begin{itemize}
            \item \textbf{Evaluation} of one person by another;
            \item \textbf{Transference} of resources (business transactions);
            \item \textbf{Association} (to social event or cause);
            \item \textbf{Behavioural} interactions (communicating);
            \item \textbf{Moving} between places or statuses (migration, social or physical mobility);
            \item Others may be: physical connection (roads, rivers), formal relations (authority), biological relationship.
        \end{itemize}
    \item \emph{Dyad} - The most basic relationship that can be established is a dyad, a connection between two actors.
    \item \emph{Triad} - A relation established between three actors. Many studies included breaking \glspl{sn} down to small groups (triads), this allowed a more clear conclusion about the transitivity of the connections.
    \item \emph{Subgroup} - It defines any subset of actors in a \gls{sn} (conceptually, subgroups come after dyads and triads).
    \item \emph{Group} - A finite set of actors who for conceptual, theoretical or empirical reasons are treated as a finite set of individuals in which network measurements are made.
    \item \emph{Relation} - A collection of ties of a specific kind among members of a group is called a \textbf{relation} (e.g. a connection in \textit{LinkedIn} is a relation while evaluating our connections of sending them messages are ties).
    \item \emph{\gls{sn}} - With the definitions of actor, group and relation, a \gls{sn} consists of a finite set or sets of actors and the relation or relations defined on them. The presence of relation information is critical and defining feature of a \gls{sn}.
\end{itemize}

%% ---------------------------------------------- Graphs Theory
\section{Graphs Theory}
Graphs are typically the base of representation of social structures. This mathematical approach maps with  extreme convenience social networks. Nodes are individuals, and edges are relationships. Despite looking a quite simple approach, there is a very strong theoretical background that is of basilar importance for interpreting social networks. In the next sections we will explore how graph theory and network analysis coexist in order to provide more formal metrics for analyzing network structures and provide information about each node within the network.

%% ---------------------------------------------- Network Analysis
\section{Network Analysis Overview}
In this section we intent to explore the scientific concepts behind network analysis, always trying to map them to reality, so only the core
and applicable concepts will be explored in this section, namely:

\begin{itemize}
\item \textbf{\textit{Power Laws}} - Power laws or power law distribution, represent in general a dependency relationship between two quantities. In \glspl{sn}, power law distribution describes a particular trend in the evolution of the number of relationships of individuals within a network;
\item \textbf{\textit{Centrality Measures}} - Centrality measures aim to answer the following question \textit{\textbf{Which vertices are important?}}. In a \glspl{sn}, an actor centrality measures the actor's interactions with other individuals;
\item \textbf{\textit{Link Analysis}} - Link analysis is a well known term form web search engines, popularized by the Page Rank algorithm. In \glspl{sn}, link analysis measures individuals connections, such as identifying strongly connected nodes, absorbing nodes or even cycles inside networks;
\item \textbf{\textit{Community Detection}} - Community detection is related to clustering in social networks. Normally when analyzing \glspl{sn} we aim for detecting communities (groups) that express similar ideas in matters such as politics, music or philosophy. Community detection is a far more abstract concept than geographical clustering, despite we often found it in \glspl{osn} such as Facebook, that the two concepts are tightly coupled;
\item \textbf{\textit{Spread of Information}} - Spread of information consists in a set of metrics that classify the propagation of the information within a network. Considering a Facebook post by a newspaper, it would come in hand to know, where was the starting point of that post, how many individuals it reaches, in which sub-networks the information was propagated, what were the entry points of for that sub-networks, how fast the information got to the individuals, these are some of the concerns relating to spread of information;
\item \textbf{\textit{Social Learning}} - Social learning consists in the change of behavior or beliefs based on direct observation of other individuals. Considering again a Facebook post by some random individual A, and consider an individual B that shares ('\textit{re-posts}') the individual's A post. If one detects a pattern in this kind of interaction, one may say that individual B is learning from individual A (imitating, mirroring).
\end{itemize}

Some of the previous listed concepts represent metrics for analyzing networks, thus requiring a more detailed explanation. In the next sections we will focus one the most fundamental metrics that will be relevant for further reference in this document. \footnote{At this point, and being network analysis basic concepts being covered it is normal that we interchangeably use the terms actor, node or vertices for denoting the same things}

%% ---------------------------------------------- Network Analysis - Fundamental Metrics
\section{Relevant metrics for network analysis}
These are crucial metrics that will be referenced within integral components of our system (that we will propose in the next Chapter 5). We will use these metrics to add value to analysis features that we will provide to the end user. For that we must first address this concepts with a smaller granularity in terms of
what they represent and also in terms of what can they offer us.

\subsection{Centrality}
Centrality if often mixed with node degree. Despite node degree being in fact used for centrality calculations, these metric have some variations
that are worth to take a close look, in order to understand the different perspectives from where we can observe a particular node in a particular network.

\subsubsection*{Degree Centrality}
\textbf{Degree} of a node is equal do the his number of adjacent nodes (or simply the number of first degree connections).
So basically what do we get from this metric? When normalized the node degree value tells us the level of direct interaction of an actor with other
actors within a network.

\subsubsection*{Closeness Centrality}
Closeness centrality tells us how close an actor is to all the other actors in the network (not only with his first degree connections).\\
\indent This metric is considered a sophisticated measure of centrality in network theory. It is defined as the mean geodesic distance
(i.e., the shortest path) between a certain vertex \textit{v} and all other vertices reachable from it. This concept is normally associated to
geographic distances, being actors closeness mapped to reality. Still there are abstractions that compute
this value not considering nodes as geodesic markers (\cite{politaktivsna}).

\subsubsection*{Betweenness Centrality}
This measure reflects the number number of shortest paths going through a particular actor. \textbf{Nodes that
occur in many shortest paths} between other nodes in the network have a higher betweenness centrality, basically
takes into account the connectivity of the nodes' neighbors, giving a \textbf{higher value for nodes which bridge clusters} (\cite{politaktivsna}).

\subsubsection*{Eigenvector Centrality}
This measure is based on the following statement:

\begin{quote}
\textbf{\textit{"Importance of a node depends on the importance of its neighbors."}}
\end{quote}

Eigenvector centrality (\cite{politaktivsna}) measures importance of a node within a network. This measure assigns relative scores to all nodes, then if
a node is connected to a \textit{high scored} node it has a bigger increment to its score then when connected to a \textit{low scored} node.
One of the most famous variants of eigenvector centrality is the Google's PageRank algorithm (\cite{brin1998anatomy}).

\subsubsection*{Page Rank}
PageRank algorithm (\cite{brin1998anatomy}) was thought as a way to rank online content (online sites) in order to discover
what sites are important and really worth to consult. The algorithm sums up a score for every node (web site),
this score is in some way proportional to the number of times a site is cross referenced (linked by other web site), gaining higher score,
on the other hand when a site links to other that has a high score it loses score, meaning that is a less important site.

\subsection{Clustering and Community Detection}
Represents the value of tendency for certain nodes to form a cluster. Normally actors within a network tend to aggregate when having some
simple characteristic in common such as living in the same city, working in the same place or event frequenting the same gymnasium.\\
\indent A common approach for detecting communities is through graph clicks (subset of vertices of an
undirected graph where all vertices are connected between each other), being the normalized clustering coefficient a high value
when the network consists in a set of disjoint clicks.

\subsection{Node Dominance}
Dominance may be related with betweeness centrality but it focus particularly on node reachability. One may say that a node \textit{\textbf{v1}} dominates
a node \textit{\textbf{v2}} if \textit{\textbf{v2}} needs to go through \textit{\textbf{v1}} to reach a certain node \textit{\textbf{v3}}.

%% ---------------------------------------------- Small World Problem
\section{Small World Problem}
This principle of \textit{small-world phenomenon} is based on the idea that all human beings are connected by \textbf{short chains of acquaintances}. The pioneer of this work was Stanley Milgram (\cite{travers1967small}).

\subsection*{Six Degrees of Separation}
The concept of six degrees of separation is an extension of the small world problem. In the sequence of what we state before, the six degrees of separation materialize the previous concept in six interconnections for some individual to reach any other one. Six degrees of separation gained a particularly strong relevance, when a play was written in the 90's portraying the concept.

\section{Network Visualization}
Network visualization may be considered as a science by itself. In the context of this project we will not look further into network visualization, we will instead
in further chapters (more technical chapters) reference advanced visualization technologies that will help us on the tool implementation serving as a fundamental complement to social network analysis.

%% ---------------------------------------------- SNA Software
\section{Social Network Analysis Software}
\label{sec:snas}

\begin{quote}
\textit{"(...) more sophisticated graphics capabilities should make exploratory studies using visual displays of networks more fruitful. One should be able to display actor attributes and nodal or subgroup properties (such as expansiveness, centrality, or clique membership) along with the graph. (...)"} \citep{wasserman1994social}
\end{quote}

Next we present some relevant software tools on \glspl{sna}.

\subsection{Structure}

\indent \indent The program Structure \citep{structure-software} is a free software package mainly used to investigate population structure. Its uses include inferring the presence of distinct populations, assigning individuals to populations, studying hybrid zones, identifying migrants and admixed individuals, and estimating population frequencies in situations where many individuals are migrants or admixed.

\subsection{Gephi}

\indent \indent Gephi \citep{bastian2009gephi} is a tool for keen data analysts and scientists who want to explore and understand graphs. Like Photoshop™ but for graph data, the user interacts with the representation, manipulates the structures, shapes and colors to reveal hidden patterns. The goal is to help data analysts to make hypothesis, intuitively discover patterns, isolate structure singularities or faults during data sourcing. It is a complementary tool to traditional statistics, as visual thinking with interactive interfaces is now recognized to facilitate reasoning. This is a software for Exploratory Data Analysis, a paradigm that appeared in the Visual Analytics field of research.

\subsection{UCINET}

\indent \indent UCINET 6 \citep{ucinet-software} for Windows is a software package for the analysis of social network data. It was developed by Lin Freeman, Martin Everett and Steve Borgatti. It comes with the \textbf{NetDraw} \citep{borgatti2002netdraw} network visualization tool.

\subsection{SocNetV}

\indent \indent Social Network Visualizer \citep{socnetv} is a cross-platform, user-friendly application for the analysis and visualization of Social Networks in the form of mathematical graphs, where vertices depict actors/agents and edges represent their relations.

With SocNetV you can construct social networks with a few clicks on a virtual canvas or load field data from various social network file formats such as GraphML, GraphViz, Adjacency, Pajek, UCINET, etc.

Furthermore, you can create random networks using various random models.

\subsection{networkx}
\indent \indent networkx \citep{hagberg2013networkx} is a Python language software package for the creation, manipulation, and study of the structure, dynamics, and functions of complex networks. networkx relevant features are listed below:
\begin{itemize}
    \item Python language data structures for graphs, digraphs, and multigraphs;
    \item Many standard graph algorithms;
    \item Network structure and analysis measures;
    \item Generators for classic graphs, random graphs, and synthetic networks;
    \item Nodes can be "anything" (e.g. text, images, XML records);
    \item Edges can hold arbitrary data (e.g. weights, time-series).
\end{itemize}

\subsection{Vizster}

\textbf{Vizster} \citep{heer2005vizster} is a tool for Visualizing online social networks, a visualization system for playful end-user exploration and navigation of large-scale online social networks.

\subsection{Project Palantir (Facebook)}

\textbf{Project Palantir} \citep{project-palantir} \footnote{This project is not at the level of \glspl{sna} specificity of the previous tools, still we consider worth mention it.} is an impressive tool that displays the rate of interactions on Facebook across the globe. This was a result of an intern annual event that happens in Facebook where all the employees are invited to participate and to build creative prototypes in the context of the company.
