%% ---------------------------------------------- Chapter Introduction

% Social network analysis is the application of network theory to the modeling and analysis of social systems.
% it combine both tools for analyzing social relations and theory for explaining the structures that emerge from the social interactions.
%
% Of course the idea of studying societies as networks is not a new one but with the rise in computation and the
% emergence of a mass of new data sources, social network analysis is beginning to be applied to all type and
% scales of social systems from, international politics to local communities and everything in between.
%
% Traditionally when studying societies we think of them as composed of various types of individuals and organizations,
% we then proceed to analysis the properties to these social entities such as their age, occupation or population, and them ascribe quantitative value to them.
%
% This allows social science to use the formal mathematical language of statistical analyst to compare the values of
%  these properties and create categories such as low in come house holds or generation x, we then search for quasi
%  cause and effect relations that govern these values.
%
% This component-based analysis is a powerful method for describing social systems. Unfortunately though is fails to
% capture the most important feature of social reality that is the relations between individuals, statistical analysis
% present a picture of individuals and groups isolates from the nexus of social relations that given them context.
%
% Thus we can only get so far by studying the individual because when individuals interact and organize, the results
% can be greater than the simple sum of its parts, it is the relations between individuals that create the emergent
% property of social institutions and thus to understand these institutions we need to understand the networks of social relations that constitute them.
%
% Ever since the emergence of human beans we have been building \glspl{sn}, we live our lives embed in networks
% of relations, the shape of these structures and where we lie in them all effect our identity and perception of the world.
%
% A social network is a system made up of a set of social actors such as individuals or organizations and a set of
% ties between these actors that might be relations of friendship, work colleagues or family. Social network science
% then analyze empirical data and develops theories to explaining the patterns observed in these networks
%
% In so doing we can begin to ask questions about the degree of connectivity within a network, its over all structure,
% how fast something will diffuse and propagate through it or the Influence of a given node within the network. lets take some examples of this
%
% Social network analysis has been used to study the structure of influence within corporations, where traditionally
% we see organization of this kind as hierarchies, by modeling the actual flow of information and communication as a
% network we get a very different picture, where seemingly irrelevant employees within the hierarchy can in fact have significant influence within the network.
%
% Researcher also study innovation as a process of diffusion of new ideas across networks, where the oval structure
% to the network, its degree of connectivity, centralization or decentralization are a defining feature in the way
% that innovation spreads or fails to spread.
%
% Network dynamics, that is how networks evolve overtime is another important area of research, for example within Law
% enforcement agencies social network analysis is used to study the change in structure of terrorists groups to identify
% changing relations through which they are created, strengthened and dissolved?
%
% Social network analysis has also been used to study patterns of segregation and clustering within international politics
% and culture, by mapping out the beliefs and values of countries and cultures as networks we can identify where opinions and beliefs overlap or conflict.
%
% Social network analysis is a powerful new method we now have that allows us to convert often large and dense data sets
% into engaging visualization, that can quickly and effectively communicate the underlining dynamics within the system.
%
% By combine new discoveries in the mathematics of network theory, with new data sources and our sociological understanding,
% social network analysis is offering huge potential for a deeper, richer and more accurate understanding, of the complex social systems that make up our world.


% RATIONALE
% As the Web rapidly evolves, Web users are evolving with it. In an era of social
% connectedness, people are becoming increasingly enthusiastic about interacting,
% sharing, and collaborating through social networks, online communities, blogs,
% Wikis, and other online collaborative media. In recent years, this collective
% intelligence has spread to many different areas, with particular focus on fields
% related to everyday life such as commerce, tourism, education, and health,
% causing the size of the Web to expand exponentially.
%
% The distillation of knowledge from such a big amount of unstructured
% information, however, is an extremely difficult task, as the contents of today’s
% Web are perfectly suitable for human consumption, but remain hardly accessible
% to machines. The opportunity to capture the opinions of the general public about
% social events, political movements, company strategies, marketing campaigns, and
% product preferences has raised growing interest both within the scientific
% community, leading to many exciting open challenges, as well as in the business
% world, due to the remarkable benefits to be had from marketing and financial
% market prediction.
%
% Existing approaches to big social data analysis mainly rely on parts of text in
% which sentiment is explicitly expressed, e.g., through polarity terms or affect
% words (and their co-occurrence frequencies). However, opinions and sentiments
% are often conveyed implicitly through latent semantics, which make purely
% syntactical approaches ineffective. In this light, this Special Issue focuses on
% the introduction, presentation, and discussion of novel techniques that further
% develop and apply affective reasoning tools and techniques for big social data
% analysis. A key motivation for this Special Issue, in particular, is to explore
% the adoption of novel affective reasoning frameworks and cognitive learning
% systems to go beyond a mere word-level analysis of natural language text and
% provide novel concept-level tools and techniques that allow a more efficient
% passage from (unstructured) natural language to (structured) machine-processable
% affective data, in potentially any domain.

\acrfull{sna} is the study of how people are connected to each other, basically it studies a set of relations among a set of entities,
these entities may be individuals, organizations, or even countries.\\\\
\indent The common analysis procedure consists in mapping the network and then creating metrics to
characterize the network. Then one tries to figure what is the structure of the network and why does
it have that structure. \glspl{sna} is also about looking at the individuals inside the network and where are those individuals located.

%% ---------------------------------------------- Fundamental Concepts for Network Analysis
\section{Fundamental Concepts for Network Analysis}

According to \cite{wasserman1994social}, the concepts listed below are of key importance to understand \glspl{sna}.

\begin{itemize}
    \item \emph{Actor} - \gls{sna} is concerned with understanding the linkages among social entities and the implications of these linkages, these social entities are described as actors. Actors are  discrete individual, corporate, or collective social units.
    \item \emph{Relational Tie} - Actors are linked to one another trough \textit{social ties}. The type of ties may be extensive, and it describes the nature of the connection. Some example of ties:
        \begin{itemize}
            \item \textbf{Evaluation} of one person by another;
            \item \textbf{Transference} of resources (business transactions);
            \item \textbf{Association} (to social event or cause);
            \item \textbf{Behavioural} interactions (communicating);
            \item \textbf{Moving} between places or statuses (migration, social or physical mobility);
            \item Others may be: physical connection (roads, rivers), formal relations (authority), biological relationship.
        \end{itemize}
    \item \emph{Dyad} - The most basic relationship that can be established is a dyad, a connection between two actors.
    \item \emph{Triad} - A relation established between three actors. Many studies included breaking \glspl{sn} down to small groups (triads), this allowed a more clear conclusion about the transitivity of the connections.
    \item \emph{Subgroup} - It defines any subset of actors in a \gls{sn} (conceptually, subgroups come after dyads and triads).
    \item \emph{Group} - A finite set of actors who for conceptual, theoretical or empirical reasons are treated as a finite set of individuals in which network measurements are made.
    \item \emph{Relation} - A collection of ties of a specific kind among members of a group is called a \textbf{relation} (e.g. a connection in \textit{LinkedIn} is a relation while evaluating our connections of sending them messages are ties).
    \item \emph{\gls{sn}} - With the definitions of actor, group and relation, a \gls{sn} consists of a finite set or sets of actors and the relation or relations defined on them. The presence of relation information is critical and defining feature of a \gls{sn}.
\end{itemize}

%% ---------------------------------------------- Graphs Theory
\section{Graphs Theory}
Graphs are typically the base of representation of social structures. This mathematical approach maps with  extreme convenience social networks. Nodes are individuals, and edges are relationships. Despite looking a quite simple approach, there is a very strong theoretical background that is of basilar importance for interpreting social networks.

%% ---------------------------------------------- Statistics
%% \section{Statistics}
%% ...

%% ---------------------------------------------- Network Analysis
\section{Network Analysis}
In this section we intent to explore the scientific concepts behind network analysis, always trying to map them to reality, so only the core
and applicable concepts will be explored in this section, namely:

\begin{itemize}
\item \textbf{\textit{Power Laws}} - Power laws or power law distribution, represent in general a dependency relationship between two quantities. In \glspl{sn}, power law distribution describes a particular trend in the evolution of the number of relationships of individuals within a network;
\item \textbf{\textit{Random Graphs}} - Radom graphs are a way of generating one of the possible graph within the universe of possibilities of a given number of nodes and a given number edges. This graphs are used to map predictions to real networks;
\item \textbf{\textit{Centrality Measures}} - Centrality measures aim to answer the following question \textit{\textbf{Which vertices are important?}}. In a \glspl{sn}, an actor centrality measures the actor's interactions with other individuals;
\item \textbf{\textit{Link Analysis}} - Link analysis is a well known term form web search engines, popularized by the Page Rank algorithm. In \glspl{sn}, link analysis measures individuals connections, such as identifying strongly connected nodes, absorbing nodes or even cycles inside networks;
\item \textbf{\textit{Community Detection}} - Community detection is related to clustering in social networks. Normally when analyzing \glspl{sn} we aim for detecting communities (groups) that express similar ideas in matters such as politics, music or philosophy. Community detection is a far more abstract concept than geographical clustering, despite we often found in \glspl{osn} such as Facebook, that the two concepts are tightly coupled;
\item \textbf{\textit{Spread of Information}} - Spread of information consists in a set of metrics that classify the propagation of the information within a network. Considering a Facebook post by a newspaper, it would come in hand to know, where was the starting point of that post, how many individuals it reaches, in which sub-networks the information was propagated, what were the entry points of for that sub-networks, how fast the information got to the individuals, these are some of the concerns relating to spread of information;
\item \textbf{\textit{Social Learning}} - Social learning consists in the change of behavior or beliefs based on direct observation of other individuals. Considering again a Facebook post by some random individual A, and consider an individual B that shares ('\textit{re-posts}') the individual's A post. If one detects a pattern in this kind of interaction, one may say that individual B is learning from individual A (imitating, mirroring).
\end{itemize}

%% ---------------------------------------------- Small World Problem
\section{Small World Problem}
This principle of \textit{small-world phenomenon} is based on the idea that all human beings are connected by \textbf{short chains of acquaintances}. The pioneer of this work was Stanley Milgram.

\subsection*{Six Degrees of Separation}
The concept of six degrees of separation is an extension of the small world problem. In the sequence of what we state before, the six degrees of separation materialize the previous concept in six interconnections for some individual to reach any other one. Six degrees of separation gained a particularly strong relevance, when a play was written in the 90's portraying the concept.

\section{Network Visualization}
Network visualization may be considered as a science by itself. In this section we will explore some relevant techniques for graph representation and visualization, we will look in particular, into \cite{beck2014state}, where a good overview is made upon graph representation approaches. We will also in a later phase of this project mentioned the graph representation technologies explored and used in this project.


%% ---------------------------------------------- Social Network Analysis Software
\section{Social Network Analysis Software}
\label{sec:snas}

\begin{quote}
\textit{"(...) more sophisticated graphics capabilities should make exploratory studies using visual displays of networks more fruitful. One should be able to display actor attributes and nodal or subgroup properties (such as expansiveness, centrality, or clique membership) along with the graph. (...)"} \citep{wasserman1994social}
\end{quote}

Next we present some relevant software tools on \glspl{sna}.

\subsection{Structure}

\indent \indent The program Structure \citep{structure-software} is a free software package mainly used to investigate population structure. Its uses include inferring the presence of distinct populations, assigning individuals to populations, studying hybrid zones, identifying migrants and admixed individuals, and estimating population frequencies in situations where many individuals are migrants or admixed.

\subsection{Gephi}

\indent \indent Gephi \citep{bastian2009gephi} is a tool for keen data analysts and scientists who want to explore and understand graphs. Like Photoshop™ but for graph data, the user interacts with the representation, manipulates the structures, shapes and colors to reveal hidden patterns. The goal is to help data analysts to make hypothesis, intuitively discover patterns, isolate structure singularities or faults during data sourcing. It is a complementary tool to traditional statistics, as visual thinking with interactive interfaces is now recognized to facilitate reasoning. This is a software for Exploratory Data Analysis, a paradigm that appeared in the Visual Analytics field of research.

\subsection{UCINET}

\indent \indent UCINET 6 \citep{ucinet-software} for Windows is a software package for the analysis of social network data. It was developed by Lin Freeman, Martin Everett and Steve Borgatti. It comes with the \textbf{NetDraw} \citep{borgatti2002netdraw} network visualization tool.

\subsection{SocNetV}

\indent \indent Social Network Visualizer \citep{socnetv} is a cross-platform, user-friendly application for the analysis and visualization of Social Networks in the form of mathematical graphs, where vertices depict actors/agents and edges represent their relations.

With SocNetV you can construct social networks with a few clicks on a virtual canvas or load field data from various social network file formats such as GraphML, GraphViz, Adjacency, Pajek, UCINET, etc.

Furthermore, you can create random networks using various random models.

\subsection{networkx}
\indent \indent networkx \citep{hagberg2013networkx} is a Python language software package for the creation, manipulation, and study of the structure, dynamics, and functions of complex networks. networkx relevant features are listed below:
\begin{itemize}
    \item Python language data structures for graphs, digraphs, and multigraphs;
    \item Many standard graph algorithms;
    \item Network structure and analysis measures;
    \item Generators for classic graphs, random graphs, and synthetic networks;
    \item Nodes can be "anything" (e.g. text, images, XML records);
    \item Edges can hold arbitrary data (e.g. weights, time-series).
\end{itemize}

\subsection{Vizster}

\textbf{Vizster} \citep{heer2005vizster} is a tool for Visualizing online social networks, a visualization system for playful end-user exploration and navigation of large-scale online social networks.

\subsection{Project Palantir (Facebook)}

\textbf{Project Palantir} \citep{project-palantir} \footnote{This project is not at the level of \glspl{sna} specificity of the previous tools, still we consider worth mention it.} is an impressive tool that displays the rate of interactions on Facebook across the globe. This was a result of an intern annual event that happens in Facebook where all the employees are invited to participate and to build creative prototypes in the context of the company.


\section{Real World Applications}
Here we will present some of the real world applications of \glspl{sna} with special attention for developed projects with a similar focus to this master's dissertation, and other tools of \glspl{osn} that cross many fields of studies. Two examples of projects with similar focus to this master's dissertation are:

\begin{itemize}
    \item \textit{\textbf{Vizster}} (\cite{heer2005vizster}) - Visualizing online social networks. A visualization system for playful end-user exploration and navigation of large-scale online social networks.
    \item \textit{\textbf{Project Palantir}} (\cite{project-palantir}) - This is an impressive tool that displays the rate of interactions on Facebook across the globe.
\end{itemize}
