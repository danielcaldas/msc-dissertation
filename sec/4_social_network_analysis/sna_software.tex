\section{Social Network Analysis Software}
\label{sec:snas}

\begin{quote}
\textit{"(...) more sophisticated graphics capabilities should make exploratory studies using visual displays of networks more fruitful. One should be able to display actor attributes and nodal or subgroup properties (such as expansiveness, centrality, or clique membership) along with the graph. (...)"} \cite{wasserman1994social}
\end{quote}

\subsection{Software Tools}
Next we present some relevant software tools on \glspl{sna}.

\subsection{Structure}

\indent \indent The program Structure \cite{structure-software} is a free software package for using multi-locus genotype data to investigate population structure. Its uses include inferring the presence of distinct populations, assigning individuals to populations, studying hybrid zones, identifying migrants and admixed individuals, and estimating population allele frequencies in situations where many individuals are migrants or admixed.

\subsection{Gephi}

\indent \indent Gephi \cite{bastian2009gephi} is a tool for keen data analysts and scientists who want to explore and understand graphs. Like Photoshop™ but for graph data, the user interacts with the representation, manipulates the structures, shapes and colors to reveal hidden patterns. The goal is to help data analysts to make hypothesis, intuitively discover patterns, isolate structure singularities or faults during data sourcing. It is a complementary tool to traditional statistics, as visual thinking with interactive interfaces is now recognized to facilitate reasoning. This is a software for Exploratory Data Analysis, a paradigm appeared in the Visual Analytics field of research.

\subsection{UCINET}

\indent \indent UCINET 6 \cite{ucinet-software} for Windows is a software package for the analysis of social network data. It was developed by Lin Freeman, Martin Everett and Steve Borgatti. It comes with the \textbf{NetDraw} \cite{borgatti2002netdraw} network visualization tool.

\subsection{SocNetV}

\indent \indent Social Network Visualizer \cite{socnetv} is a cross-platform, user-friendly application for the analysis and visualization of Social Networks in the form of mathematical graphs, where vertices depict actors/agents and edges represent their relations.

With SocNetV you can construct social networks with a few clicks on a virtual canvas or load field data from various social network file formats such as GraphML, GraphViz, Adjacency, Pajek, UCINET, etc.

Furthermore, you can create random networks using various random models.

\subsection{NetworkX}
\indent \indent NetworkX \cite{hagberg2013networkx} is a Python language software package for the creation, manipulation, and study of the structure, dynamics, and functions of complex networks. NetworkX relevant features are listed below:
\begin{itemize}
    \item Python language data structures for graphs, digraphs, and multigraphs;
    \item Many standard graph algorithms;
    \item Network structure and analysis measures;
    \item Generators for classic graphs, random graphs, and synthetic networks;
    \item Nodes can be "anything" (e.g. text, images, XML records);
    \item Edges can hold arbitrary data (e.g. weights, time-series).
\end{itemize}

\subsection{Vizster}

\textbf{Vizster} (\cite{heer2005vizster}) is a tool for Visualizing online social networks, a visualization system for playful end-user exploration and navigation of large-scale online social networks.

\subsection{Project Palantir (Facebook)}

\textbf{Project Palantir} (\cite{project-palantir}) \footnote{This project is not at the level of \glspl{sna} specificity of the previous tools, still we consider worth mention it.} is an impressive tool that displays the rate of interactions on Facebook across the globe. This was a result of an intern annual event that happens in Facebook where all the employees are invited to participate and to build creative prototypes in the context of the company.
