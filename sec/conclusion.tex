In this Chapter we look at our work retrospectively and we resume what we have achieved and we do a work summarization of our contribution with this
dissertation.\\
\indent At the very start of our work we had limited expectations since the definition of the final product itself it's part of the work we have done. We first start for studying how \glspl{sn} came to exist and in what ways we first perceive them (Chapter 2), only after realizing the time that sociology had already invested in this subject, only then we start investigating the \glspl{osn} that we described as the manifestation of \glspl{sn} of our time, the Internet. Many \glspl{osn} were deeply analyzed in Chapter 3, we looked into on how they are composed and what drives users to use them.\\
\indent Then we needed to know how this social structures are studied and interpreted from a scientific perspective, this set us in the way of \glspl{sna} (Chapter 4). With \glspl{sna} we are able to map social structures in mathematic abstractions, the first step always consists in representing the network by means of a graph, from there on \glspl{sna} field has already some well established metrics such as centrality or clustering coefficient that depending in what we want perceive help us to discover a series of facts about a network, such as how influent are individuals, how many communities there are, is the network dense, and all the others that we state in Chapter 4.\\
\indent From here we started to design our solution in order to define a useful tool that would help users understanding their social structures in these \glspl{osn}. In Chapter 5 we presented our solution in the most conceptual way, in Chapter 6 we defined the requirements and their respective priorities in order to obtain a minimum viable product at the end of the project.\\
\indent At the same time that we were developing our tool we documented all the relevant technicalities in Chapter 7. Finally having Socii been implemented and tested we present all the final results in Chapter 8, this chapter contains a \textit{walktrough} of the functionalities of our tool, it also includes case studies that demonstrate how the end user can use Socii to obtain concrete results, with these we prove the utility of our final product.

%% ---------------------------------------------- The main obstacle for Socii
\section{The main obstacle for Socii}
As he have seen trough the thesis since the beginning we based our work on \glspl{osn}, and we have seen as well that our platform is data driven meaning that it is built on the assumption of available and accessible data, but in reality this is not happening, \glspl{osn} has we have seen in Chapter 3 are not \textit{"opening the doors"} to the community making their \glspl{api} and social public data available, that is why we went trough the technical and architectural struggle of feeding in Socii networks trough a extraction pipeline built on top of web crawlers, that are known and probed to be very slow and error prune. If \glspl{osn} such as Facebook or LinkedIn provide access to their social \glspl{api} Socii final results could be much more positive and surprising.

\subsection*{Other ways on how \glspl{osn} could contribute}
In fact yes, despite these platforms could open their \glspl{osn} to the public they should also play an extremely important role in social education, security, environment (etc.), they have today the power to influence their users in good ways.

%% ---------------------------------------------- Alternative technical approaches that could improve Socii
\section{Alternative technical approaches that could improve Socii}
In this section we will explore alternative approaches that can bee implemented in order to improve certain bottlenecks of Socii such as performance. We will list this alternatives explaining both what Socii could gain and loose by selecting those paths. Projects such as \citep{graphosaurus} would be helpfull on this implementation.

\subsection{Visualization}

\subsection*{Using WebGL for network visualization and interaction}
Web GL \citep{marrin2011webgl} could have been the future of Socii if instead of a two dimensions network representation we have choose to go on to the third dimension. This would resolve the node overlapping node and could make the network discovery process a more simple process, since nodes would have more space to rearrange themselves.

\subsection{Performance}

\subsection*{Using server side rendering}
Server side rendering is a technique where visual components (templating work) is done in the server side, this normally bring to web applications improvements in terms of spent time in rendering and building templates, work that is usually done by the client according to architectural definition of more recent front end frameworks and libraries.\\
\indent Server side rendering, in our specific case could be a good approach since all the heavy calculations for positioning nodes
may be done on the server instead of being done on the browser, this would however have impacts in terms of scalability if we had to many users requesting
the rendering of huge networks.

\subsection*{Using web workers for heavy front end background processing}
Modern browsers are close to fully support all the HTML5 new features, this including web workers \citep{webworkers}. This new technologies allows
the browser to run a script operation in background thread separate from the main execution thread of a web application \citep{mdnwebworkers}. For Socii it would be very helpful to have some place where to run some calculations as asynchronous tasks, this could for example to our metrics calculations instead of the current approach where we need to make an http request to the metrics microservice in order to fetch network metrics.

%% ---------------------------------------------- Socii usage and applications
\section{Socii usage and applications}
We have already demonstrated some case studies in Chapter 8 where we demonstrated some of the potential utilities of our tool. In this section we will meditate and speculate upon Socii potential of usage across several fields of study. So what could be Socii real applications?

\begin{itemize}
    \item \textbf{Sociology general studies, social analysis} - Basically where Socii is used merely as a \glspl{sna} tool used by scientist and students of the field.
    \item \textbf{Migratory flux of population} - Having a tool such as Socii that allows us to have a macroscopic overview upon social networks we could study population migratory flux (using community detection for example) to understand what is the shape and trends of population migration across the globe, at the time of this writing this could be helpful for example on the detection of refugees communities where we could detect what commutes where formed with existing and stable communities of other countries and how this affects both refugees and the local population;
    \item \textbf{Society happiness studies/Depression detection among youngsters} - We could use Socii to detect cases of depression among youngsters, this is today
    unfortunately a very common disease that urges among young people and that could be prevented by monitoring social networks usage among this youngsters and being alter
    for strange/abnormal behaviors;
    \item \textbf{Terrorism awareness/detection} - Using a similar strategy to the one we used to detect refugees communities we could analyze data and look for strange patters of interactions concerning individuals origins. A simple example could be, and individual with nationality X that belongs to a normal network and suddenly starts to create online connections with individuals of the nationality Y and this nationality is blacklisted as possible association with terrorism, we could see this as a potential threat;
    \item \textbf{Marketing} - As we seed in Chapter 8 one of the use cases was marketing, we could use Socii to detect potential target audiences for a given brand, product or service.
\end{itemize}

%% ---------------------------------------------- Alternative work
\section{Alternative work}
From analyzing various \glspl{osn} in Chapter 3 we have seen a possible research project on creating a framework for building
and managing \glspl{osn} automatically and effortlessly. This framework could allow \glspl{osn} to be created on the fly with a model based approach, where the user/programmer
just need to insert a model and the \glspl{osn} would be generated.

%% ---------------------------------------------- Future work
\section{Future work}
In Chapter 6 we already described a lot of the future work that could be done regarding Socii, all the requirements that were not marked as \textbf{MUST} requirements
are improvements to the Socii tool that we see as relevant future work. Other future work would be:

\begin{itemize}
    \item Improve network extraction process and allow users to build their networks on the fly;
    \item Adapt the current approach that we have today on Socii that builds social structures based on individuals relations to do the same thing with terms/keywords
    building a network of co-related keywords within a restrict domain/theme;
    \item If eventually \glspl{osn} make their social \glspl{api} available migrate from using Socii web crawlers to consuming directly this \glspl{api}. This would
    considerably increase user experience and allow us to fulfill the first item of the \textit{future work list} that is to allow users to quickly build their networks.
    \item Understand better Socii positioning among the social analytics world and try to find new and innovative applications for this tool;
\end{itemize}

% For presentation day think of this things
% - Is this a OAIS system? Study this because they may ask!
% - Who are the people that will be evaluating me and what are their fields of studies?