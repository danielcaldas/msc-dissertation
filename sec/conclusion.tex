In this Chapter we look at our work retrospectively and we resume what we have achieved and we do a work summarization of our contribution with this
dissertation.\\
\indent At the very start of our work we had limited expectations since the definition of the final product itself it's part of the work we have done. We first start for studying how \glspl{sn} came to exist and in what ways we first perceive them, only after realizing the time that sociology had already invested in this subject, only then we start investigating the \glspl{osn}.


With our research and development we can certainly prove that there is room for growth and propagation of social analytics tools among the online world. Despite very advandced tools are developer .... desktop...
\\
We study...\\
We developed...\\
We proved...\\

These represent topics to not forget to mention on future work and conclusions:
\begin{itemize}
    \item Web GL is the better option if we want to aim for performance on the web;
    \item From analyzing OSN we see a possible research project on creating a framework for building
    and managing \glspl{osn} automatically and effortlessly;
    \item Server side rendering may also be a good approach since all the heavy calculations for positionnig nodes
    may be donne in the server side, this would however have impacts in terms of scalability if we had to many users requesting
    the renderization of huge networks.
\end{itemize}

%% Why socci? What is main purpose? How can we observe it from the built functional prototype
Expand horizons... What fields could we explore...\\
\begin{itemize}
    \item Sociology general studies, society analysis;
    \item Migratory flux of population (maybe even mention the refugees crisis);
    \item Society happiness studies/Depression detection among youngsters;
\end{itemize}
% Talk here about Socci applications where it could be used...!!!
% - Terrorism awareness detection (exemplo: analisar relações dos individuos comparativamente às suas nacionalidades.. se
% de repente tenho um individuo de nacionalidade X que nunca falou com ninguem da nacionalidade Y e está agora a falar com 20 individuos da nacionalidade Y lançar um alerta!);
% - Is this a OAIS system? Study this because they may ask!
% - Parallel studies - framework to create Online Social Networks Platforms, this could be huge.. Another theis... Other branch of studies
% of computer science + social sciences
