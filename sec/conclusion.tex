In this Chapter we look at our work retrospectively and we resume what we have achieved and we do a work summarization of our contribution with this
dissertation.\\
\indent At the very start of our work we had limited expectations since the definition of the final product itself it's part of the work we have done. We first start for studying how \glspl{sn} came to exist and in what ways we first perceive them (Chapter 2), only after realizing the time that sociology had already invested in this subject, only then we start investigating the \glspl{osn} that we described as the manifestation of \glspl{sn} of our time, the Internet. Many \glspl{osn} were deeply analyzed in Chapter 3, we looked into on how they are composed and what drives users to use them.\\
\indent Then we needed to know how this social structures are studied and interpreted from a scientific perspective, this set us in the way of \glspl{sna} (Chapter 4). With \glspl{sna} we are able to map social structures in mathematic abstractions, the first step always consists in representing the network by means of a graph, from there on \glspl{sna} field has already some well established metrics such as centrality or clustering coefficient that depending in what we want perceive help us to discover a series of facts about a network, such as how influent are individuals, how many communities there are, is the network dense, and all the others that we state in Chapter 4.\\
\indent From here we started to design our solution in order to define a useful tool that would help users understanding their social structures in these \glspl{osn}. In Chapter 5 we presented our solution in the most conceptual way, in Chapter 6 we defined the requirements and their respective priorities in order to obtain a minimum viable product at the end of the project.\\
\indent At the same time that we were developing our tool we documented all the relevant technicalities in Chapter 7. Finally having Socii been implemented and tested we present all the final results in Chapter 8, this chapter contains a \textit{walktrough} of the functionalities of our tool, it also includes case studies that demonstrate how the end user can use Socii to obtain concrete results, with these we prove the utility of our final product.

%% ---------------------------------------------- The main obstacle for Socii
\section{The main obstacle for Socii}
As he have seen trough the thesis since the beginning we based our work on \glspl{osn}, and we have seen as well that our platform is data driven meaning that it is built on the assumption of available and accessible data, but in reality this is not happening, \glspl{osn} has we have seen in Chapter 3 are not \textit{"opening the doors"} to the community making their \glspl{api} and social public data available, that is why we went trough the technical and architectural struggle of feeding in Socii networks trough a extraction pipeline built on top of web crawlers, that are known and probed to be very slow and error prune. If \glspl{osn} such as Facebook or LinkedIn provide access to their social \glspl{api} Socii final results could be much more positive and surprising.

\subsection*{Other ways on how \glspl{osn} could contribute}
In fact yes, despite these platforms could open their \glspl{osn} to the public they should also play an extremely important role in social education, security, environment (etc.), they have today the power to influence their users in good ways.

%% ---------------------------------------------- Alternative technical approaches that could improve Socii
\section{Alternative technical approaches that could improve Socii}
In this section we will explore alternative approaches that can bee implemented in order to improve certain bottlenecks of Socii such as performance. We will list this alternatives explaining both what Socii could gain and loose by selecting those paths.

\subsection{Using WebGL for network visualization and interaction}
Web GL REF is the better option if we want to aim for performance on the web when display heavy graphical elements on the browser. This would require an adaptation of our front end app in order to couple this \textit{"foreign technologie"} since our front end stack is built in Javascript only. Dispite the performance gains we would end up with a much more complex tool (and Socii front end has already is share of complexity), also harder to mantain and to scale in terms of new functionalities.

\subsection{Using server side rendering}
Server side rendering may also be a good approach since all the heavy calculations for positioning nodes
may be done in the server side, this would however have impacts in terms of scalability if we had to many users requesting
the rendering of huge networks.

\subsection{Using web workers for heavy front end background processing}
Modern browsers already support web workers REF...

%% ---------------------------------------------- Socii usage and applications
\section{Socii usage and applications}
We have already demonstrated some case studies/use cases in Chapter 8... still..
Explore where Socii could be used... What are the real world applications more suitable for this tool
Expand horizons... What fields could we explore...\\
\begin{itemize}
    \item Sociology general studies, society analysis;
    \item Migratory flux of population (maybe even mention the refugees crisis);
    \item Society happiness studies/Depression detection among youngsters;
    \item Terrorism awareness detection (exemplo: analisar relações dos individuos comparativamente às suas nacionalidades.. se
    de repente tenho um individuo de nacionalidade X que nunca falou com ninguem da nacionalidade Y e está agora a falar com 20 individuos da nacionalidade Y lançar um alerta!.
\end{itemize}

%% ---------------------------------------------- Alternative work
\section{Alternative work}
From analyzing OSN we see a possible research project on creating a framework for building
and managing \glspl{osn} automatically and effortlessly;
Parallel studies - framework to create Online Social Networks Platforms, this could be huge.. Another theis... Other branch of studies

%% ---------------------------------------------- Future work
\section{Future work}
All the requirements in Chapter 6 that were not relevant for the purpose of this master's dissertation could bring many improvements to the existing tool...

%% ---------------------------------------------- Final note
\section{Final note}
\textbf{Mencionar a publicação no SLATE' 17 ????}
\textbf{Mencionar que fizemos validação de trabalho e recolha de feedback com alguém da área ???}

% For presentation day think of this things
% - Is this a OAIS system? Study this because they may ask!
% - Who are the people that will be evaluating me and what are their fields of studies?