Until now, the work developed under this master thesis project was manly research, we not only walked trough concepts in sociology that are of basilar importance
for fundament and building the context to the problem we presented. Then we start focusing in information technology and presented the \glspl{osn}, exploring them
with some depth. Then, we present some theory of \glspl{sna} with focus in what we want to implement.

\section{Working Plan}

In this section is define the working plan for this master's thesis project, with the aim of structuring the project approach. Next, follows a list of the steps for this project.

\begin{itemize}
\item $1^{st}$ \textbf{Research} - This first phase will be a exploratory overview on the theory of social networks, social network analysis and graphs visualization. It will also be a phase to study the state of the art of
social networks analysis and visualization tools.
\item $2^{nd}$ \textbf{Requirements Specification} - Here, the system must be specified. This includes well defining the proposed system, as well as the features that the system will provide to the end user. The requirements should also establish a range of social networks that the system could support.
\item $3^{rd}$ \textbf{System Modelling} - In this phase the focus goes to the system architecture design, conceptual data model and high level graphical interface mock ups.
\item $4^{th}$ \textbf{Choice of Technologies} - In the 4th phase the technologies to implement the previous specified system must be chosen. It will be an opportunity to explore the
most promising technologies implementing some small prototypes, merely as proof of concept.
\item $5^{th}$ \textbf{System Implementation} - The implementation phase should be iterative, a divide and conquer strategy will be followed to break the system into small pieces so it grows the more naturally allowing
coherence and robustness in the software.
\item $6^{th}$ \textbf{Evaluation and Conclusions} - For last the system must be evaluated and analyzed to see if it fits the purpose for which it was build and if it fills the pre-established requirements. Concluding, a retrospective of the work must be done, such as a global overview on the developed system and some final comments on the obtained results.
\end{itemize}

\section{Where we are and Future Work}

At this point, the first phase is almost completed, missing only some more depth in the study of some concepts of \glspl{sna} in chapter 4.\\
\indent The requirements specification has already started by the study and proposal of the architecture in the previous chapter. We may now proceed to the detail and formal definition of the requirements and refine the already proposal architecture. From here, we will follow the steps of the working plan that consist in choosing the suitable technologies, implementing the tool, test it, deriving conclusions and documenting the obtained results.
