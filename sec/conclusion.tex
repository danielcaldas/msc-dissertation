In this Chapter we look at our work retrospectively and we resume what we have achieved and we do a work summarization of our contribution with this
dissertation.\\
\indent At the very start of our work we had limited expectations since the definition of the final product itself it's part of the work we have done. We first start for studying how \glspl{sn} came to exist and in what ways we first perceive them (Chapter 2), only after realizing the time that sociology had already invested in this subject, only then we start investigating the \glspl{osn} that we described as the manifestation of \glspl{sn} of our time, the Internet. Many \glspl{osn} were deeply analyzed in Chapter 3, we looked into on how they are composed and what drives users to use them.\\
\indent Then we needed to know how this social structures are studied and interpreted from a scientific perspective, this set us in the way of \glspl{sna} (Chapter 4). With \glspl{sna} we are able to map social structures in mathematic abstractions, the first step always consists in representing the network by means of a graph, from there on \glspl{sna} field has already some well established metrics such as centrality or clustering coefficient that depending in what we want perceive help us to discover a series of facts about a network, such as how influent are individuals, how many communities there are, is the network dense, and all the others that we state in Chapter 4.\\
\indent From here we started to design our solution in order to define a useful tool that would help users understanding their social structures in these \glspl{osn}. In Chapter 5 we presented our solution in the most conceptual way, in Chapter 6 we defined the requirements and their respective priorities in order to obtain a minimum viable product at the end of the project.\\
\indent At the same time that we were developing our tool we documented all the relevant technicalities in Chapter 7. Finally having Socii been implemented and tested we present all the final results in Chapter 8, this chapter contains a \textit{walktrough} of the functionalities of our tool, it also includes case studies that demonstrate how the end user can use Socii to obtain concrete results, with these we prove the utility of our final product.

\section{Could big \glspl{osn} contribute to this field}
YES. MAINLY THEM. Open data, let developers query public data freely.

\section{Alternative technical approaches that could improve Socii}
...
...
These represent topics to not forget to mention on future work and conclusions:
\begin{itemize}
    \item Web GL REF REF is the better option if we want to aim for performance on the web;
    \item From analyzing OSN we see a possible research project on creating a framework for building
    and managing \glspl{osn} automatically and effortlessly;
    \item Server side rendering may also be a good approach since all the heavy calculations for positioning nodes
    may be done in the server side, this would however have impacts in terms of scalability if we had to many users requesting
    the rendering of huge networks.
\end{itemize}

\section{Socii usage and applications}
%% Why socci? What is main purpose? How can we observe it from the built functional prototype
Expand horizons... What fields could we explore...\\
\begin{itemize}
    \item Sociology general studies, society analysis;
    \item Migratory flux of population (maybe even mention the refugees crisis);
    \item Society happiness studies/Depression detection among youngsters;
\end{itemize}
% Talk here about Socci applications where it could be used...!!!
% - Terrorism awareness detection (exemplo: analisar relações dos individuos comparativamente às suas nacionalidades.. se
% de repente tenho um individuo de nacionalidade X que nunca falou com ninguem da nacionalidade Y e está agora a falar com 20 individuos da nacionalidade Y lançar um alerta!);
% - Is this a OAIS system? Study this because they may ask!
% - Parallel studies - framework to create Online Social Networks Platforms, this could be huge.. Another theis... Other branch of studies
% of computer science + social sciences
\section{Derived work??????????}

\section{Final note}
\textbf{Mencionar a publicação no SLATE' 17 ????}
\textbf{Mencionar que fizemos validação de trabalho e recolha de feedback com alguém da área ???}